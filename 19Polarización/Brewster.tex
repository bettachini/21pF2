\section*{Ángulo de Brewster}

\item Incide un haz de luz linealmente polarizada sobre la superficie de separación de dos medios transparentes.
¿Qué condiciones deben cumplirse para que ese haz se transmita totalmente hacia el segundo medio?



\item Un haz de luz circularmente polarizada en sentido horario incide con el ángulo de polarización sobre la superficie de separación de dos medios transparentes.
¿Cuál es el estado de polarización del haz reflejado?
¿Y del transmitido?
Justifique.



\item Sobre una superficie de separación entre dos medios de índices $n_1$ y $n_2$ (con $n_1 > n_2$), incide un rayo desde el medio $n_1$.
\begin{enumerate}
	\item ¿Cuál es el ángulo de incidencia crítico a partir del cual se produce reflexión total? 
	\item ¿Cuál es el ángulo de polarización? 
	\item ¿Es posible que el ángulo de polarización sea mayor que el ángulo crítico?
	Justifique físicamente y analíticamente.
\end{enumerate}



\item Sobre una lámina de vidrio de caras paralelas y de índice $n$, colocada en aire, se hace incidir luz elípticamente polarizada con el ángulo de polarización.
Se analiza el haz reflejado.
\begin{enumerate}
	\item ¿Cuál es su estado de polarización?
	\item Ahora, sin modificar la dirección del haz incidente, se sumerge a la lámina parcialmente en agua, de forma tal que sobre la cara superior hay aire.
	¿Cuál es el estado de polarización del haz reflejado?
	\item Ahora se sumerge la lámina totalmente en el agua, sin modificar la dirección del haz que incide sobre la lámina.
	¿Cuál es el estado de polarización del haz reflejado?
	¿Cómo podría lograr que la polarización del haz reflejado fuera linealmente polarizado?
\end{enumerate}
