\section*{Parámetros de una onda propagante}


\item Verifique si las siguientes expresiones cumplen la ecuación de las ondas unidimensional.
Grafíquelas.
\begin{tasks}(3)
	\task $\psi(x,t)= A \operatorname{e}^{- \lambda ( x - v t)^2 }$
	\task $\psi(x,t)= \beta ( x + v t )$
	\task $\psi(x,t)= A \sen{\left[ k (x - v t) \right] }$
	\task $\psi(x,t)= B \sen^2{ \left( k x -\omega t \right) }$
	\task $\psi(x,t)= C \cos{(k x)} \sen{ \left( \omega t \right) }$
	\task $\psi(x,t)= D \operatorname{e}^{i ( k x - \omega t ) }$
\end{tasks}



\item Una onda se propaga en una cuerda produciendo una oscilación transversal que responde a 
\[
\psi(x,t) = \SI{0.1}{\metre} \sen \left( \SI{\pi}{\per\metre} x - \SI{4 \pi}{\per\second} t \right).
\]
Determine:
\begin{tasks}(2)
	\task amplitud,
	\task frecuencia de vibración, y
	\task velocidad de propagación.
	\task $x = \SI{2}{\metre}$ y $ t = \SI{1}{\second}$, desplazamiento, velocidad y la aceleración de la cuerda.
\end{tasks}



\item Una onda de $\omega= \SI{10}{\per\second}$ se propaga en $\hat{x}$ con $k = \SI{100}{\per\metre}$.
En $x_1 = \SI{1}{\kilo\metre}$ y $t_1 = \SI{1}{\second}$ tiene por fase $\phi = \frac{3 \pi}{2}$.
\begin{tasks}
	\task ¿Cuál es la fase en ese mismo punto para $t = 0$?
	\task Considerando que $\phi(x,t) = k x - \omega t+ \phi_0$, ¿cuánto vale $\phi_0$?
	\task ¿A qué velocidad se propaga la onda?
	\task ¿En qué tiempo el frente de onda arriba a un $x_2 = 2 x_1$?
\end{tasks}



\item Una cuerda de densidad lineal $\mu = \SI{0.005}{\kilo\gram\over\metre}$ se tensa con una fuerza de \SI{0.25}{\newton}.
Se observa que un punto se verifica la máxima amplitud víendose entre el extremo arriba y el que alcanza abajo un distanciamiento de \SI{0.4}{\metre}.
Entre alcanza un extremo y otro transucurren \SI{0.25}{\second}.
Encontrar:
\begin{enumerate}
	\item La velocidad de la onda generada en la cuerda, la frecuencia y la longitud de onda.
	\item La expresión matemática para el desplazamiento: $\psi(x,t)$.
	\item La energía cinética media por unidad de longitud, de una partícula del medio.
	\item La energía potencial media por unidad de longitud, de una partícula.
\end{enumerate}
