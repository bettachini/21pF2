\subsection*{Difracción en N=2 rendijas}

\item 
\begin{enumerate}
	\item Se tienen dos rendijas iguales, de ancho $b$, cuya separación entre centros es $d$, colocadas entre dos lentes delgadas convergentes, ubicadas en forma simétrica respecto del eje óptico del sistema.
	Una fuente puntual monocromática que emite con $\lambda$ se encuentra en el foco de la primera lente. Considere la figura de interferencia--difracción de Fraunhofer de la fuente. 
	\begin{enumerate}
		\item Calcule la posición de los máximos y mínimos tanto de interferencia como de difracción. 
		\item Grafique la irradiancia sobre la pantalla, ¿en función de qué variable lo hace?
		¿Qué otra variable podría haber usado?
		\item Suponiendo que la teoría fuese exacta, ¿qué condiciones deberían cumplirse para que desaparezcan órdenes, y cuáles serían los órdenes desaparecidos? 
		\item ¿Cuántos órdenes de interferencia hay dentro de la campana principal de difracción?
		\item A la luz de estos resultados discuta el interferómetro de Young. 
		\item Considere que la fuente emite en $\lambda$, $2\lambda$ y $3\lambda$ simultáneamente.
		Para cada una de dichas longitudes de onda, ¿cuál es la posición de los máximos y mínimos de interferencia y difracción?
		En particular, ¿cuál es la posición del máximo principal?
	\end{enumerate}
	\item Repita lo hecho en (a), si la fuente se encuentra a una altura $h$ del eje óptico. 
	\item Idem (b) si el punto medio entre ranuras se encuentra a una altura $h'$ del eje óptico. 
\end{enumerate}


\item Se realiza una experiencia de difracción por doble rendija con una fuente que emite en \SI{4000}{\angstrom}.
La separación entre los puntos medios de las rendijas es de \SI{0.4}{\milli\metre} y el ancho de cada una de ellas es de \SI{0.04}{\milli\metre}.
La pantalla está a \SI{1}{\metre} de las rendijas.
Luego se cambia la fuente por otra que emite en \SI{6000}{\angstrom}.
Determine:
\begin{enumerate}
	\item En cuánto varió la interfranja. 
	\item En cuánto varió el número total de franjas de interferencia contenidas en la campana principal de difracción. 
	\item En cuánto varió el ancho angular de la campana principal de difracción. 
\end{enumerate}



\item Sobre dos ranuras separadas una distancia de \SI{1}{\milli\metre} incide la superposición de dos ondas planas monocromáticas de longitudes de onda $\lambda_1$ y $\lambda_2$
\begin{enumerate}
	\item ¿Qué relación debe satisfacer el cociente $\lambda_1/\lambda_2$ para que el tercer orden de interferencia constructiva de $\lambda_1$ coincida con el tercer mínimo de $\lambda_2$? 
	\item ¿Qué ancho deben tener las ranuras para que además esos órdenes coincidan con el primer mínimo de difracción de $\lambda_1$?
	¿Qué irradiancia se registrará en la pantalla en ese punto? 
\end{enumerate}
