\section*{Red de difracción de surcos ultrafinos (\emph{diffraction grating})}



\item \textbf{Red por transmisión}
Se dispone de dos redes de difracción cuadradas de \SI{2}{\centi\metre} de lado.
La densidad de líneas de una es \SI{600}{\per\milli\metre} y la de otra \SI{1200}{\per\milli\metre}.
Calcule: 
\begin{enumerate}
	\item El poder resolvente de cada red en el primer orden. 
	\item El máximo orden observable con una fuente de \SI{5000}{\angstrom}. 
	¿Es importante tener en cuenta el ángulo de incidencia? 
	\item El máximo poder resolvente de cada una. 
	\item Si alguna de ellas resuelve entre $\lambda_1 = \SI{5000}{\angstrom}$ y $\lambda_2 = \SI{5000,07}{\angstrom}$.
\end{enumerate}




\item (*) 
\textbf{Red por reflexión}
Se desea estudiar la estructura de una banda en la proximidad de \SI{4300}{\angstrom}, utilizando una red plana de reflexión de \SI{10}{\centi\metre} y una densidad de líneas de \SI{1200}{\per\milli\metre}.
Hallar:
\begin{enumerate}
	\item El máximo orden observable. 
	\item El mínimo ángulo de incidencia para el cual se observa. 
	\item El mínimo intervalo de longitudes de onda resueltas. 
	\item El orden intensificado.
	¿Es ventajoso?
	Justifique su respuesta. 
\end{enumerate}
