\section*{Prisma}

\item
\begin{enumerate}
	\item 
	Tras atravesar un prisma un haz monocromático se desvía \(\delta \alpha\).
	Haga un gráfico cualitativo que le permita ver el efecto sobre \(\delta \alpha\) del ángulo de incidencia, \(\theta_{i1}\).
	Para esto dibuje dos casos extremos.
	\item Calcule el ángulo de desviación mínima \(\delta \alpha_\text{mín}\).
\end{enumerate}


\item
\begin{enumerate}
	\item En un vidrio se propaga un haz de luz blanca.
	¿Qué componente viaja más rápido, la roja o la violeta?
	\item ¿Cuál de estos tiene mayor desviación?
	% ¿Qué puede decir del ángulo de desviación mínima \(\delta \alpha_\text{mín}\)?
\end{enumerate}



\item 
Calcular las desviaciones de rayos que inciden casi perpendicularmente de las \emph{líneas de absorción de Fraunhofer} F, D\textsubscript{1} y C, en un prisma de \emph{vidrio Crown} de ángulo $\alpha = \ang{4;;}$.
Los respectivos índices son: $n_F = 1.513$; $n_D = 1.508$ y $n_C =1.504$.
