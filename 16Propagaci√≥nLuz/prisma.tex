\section*{Prisma}

\item
\begin{enumerate}
	\item Un haz de luz monocromático al atravesar un prisma cambia su dirección en un ángulo de desviación \(\delta \alpha\).
	Haga un gráfico cualitativo que le permita ver el efecto sobre \(\delta \alpha\) del ángulo con que incidió en las cara que numeraremos 1.
	Para esto dibuje dos casos de ángulos respecto a la normal de esa cara \(\theta_{i1}\) bien distintos.
	\item Calcule analíticamente el ángulo de desviación mínima \(\delta \alpha_\text{mín}\) para un prisma en función de los datos constructivos.
\end{enumerate}


\item
\begin{enumerate}
	\item En un vidrio óptico común se propaga un haz de luz blanca, ¿qué componente viaja más rápido: la roja o la violeta?
	\item ¿Para cuál de ambos colores será mayor la desviación en un prisma \(\delta \alpha_n\) ?
	¿Qué puede decir del ángulo de desviación mínima \(\delta \alpha_\text{mín}\)?
\end{enumerate}



\item Dado un prisma de \emph{vidrio Crown} de ángulo $\alpha = \ang{4;;}$ calcular, para las \emph{líneas de absorción de Fraunhofer} F, D\textsubscript{1} y C, las desviaciones de rayos que inciden casi perpendicularmente.
Los respectivos índices son: $n_F = 1.513$; $n_D = 1.508$ y $n_C =1.504$.
