\section*{División de frente de onda vs. división de amplitud}

\item Haga un cuadro comparativo de las magnitudes que caracterizan a los distintos interferómetros por división de frente de onda e indique en cada uno de ellos cómo se divide el frente. 


\item Indique en cada uno de los interferómetros por división de amplitud estudiados dónde se divide la amplitud.
¿Son iguales las amplitudes de los haces que interfieren? En la lámina de caras paralelas compare estas amplitudes tanto en la salida por reflexión como por transmisión para incidencia normal.


\item Diga qué entiende por interferómetro por división de amplitud.
Enumere los más representativos e indique en un esquema sus parámetros característicos. 


\item ¿Qué entiende por franjas localizadas de interferencia?
¿En qué casos están localizadas las franjas en un interferómetro por división de amplitud?
Justifique sus respuestas. 
