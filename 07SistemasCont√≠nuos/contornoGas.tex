\subsection*{Condiciones de contorno para el gas en un tubo unidimensional}

\item \textbf{Tubo con aire}
En un tubo de longitud $L$ hay aire sin humedad.
Considere las siguientes posibilidades: 
\begin{tasks}(3)
	\task cerrado en ambos extremos,
	\task uno abierto el otro cerrado, y
	\task ambos extremos abiertos.
\end{tasks}
Halle para cada una de dichas situaciones: 
\begin{enumerate}
	\item Las posibles longitudes de onda con las que puede vibrar el aire en el tubo, y sus correspondientes frecuencias. 
	\item Elija un sistema de referencia conveniente, y escriba la expresión más general para el desplazamiento de las partículas $\Psi(x,t)$.
	¿Qué parámetros conoce de dicha expresión?
	¿De qué dependen los que no conoce? 
	\item A partir de la expresión de $\Psi(x,t)$, hallar el apartamiento de la presión respecto a la atmosférica $\delta p(x,t)$.
	¿Cuál es la diferencia de fase entre ellas?
	¿Cuál es la amplitud máxima de presión? 
	\item Hallar la función de densidad $\rho(x,t)$.
	¿Cuál es amplitud máxima?
\end{enumerate}
Asuma conocidos: $L$, $v_\text{sonido}$, la presión atmosférica $p_0$, $\rho_{0}= \frac{ \gamma p_0 }{ v_\text{sonido}^2}$ y que $\gamma= \frac{C_p}{C_v}= \frac{7}{5}$ para un gas diatómico (¿Cuales son las dos moléculas preponderantes en la atmósfera?).


\item \textbf{Tubo de órgano}. 
\begin{enumerate}
\item ¿Qué longitud debe tener un tubo de órgano de extremos abiertos para producir un sonido de \SI{440}{\hertz}? 
\item Si uno de sus extremos está cerrado y se desea producir el mismo tono en su primer armónico, ¿qué longitud deberá tener?
\end{enumerate}


\item Se tiene un tubo cerrado en uno de sus extremos; su longitud es menor a \SI{1}{\metre}.
Se acerca al extremo abierto un diapasón que está vibrando con $\nu = \SI{440}{\hertz}$.
Considere $v_\text{sonido} = \SI{330}{\metre\over\second}$.
\begin{enumerate}
	\item Hallar las posibles longitudes del tubo para que haya resonancia.
	Para cada una de ellas, ¿en qué modo está vibrando el aire contenido en el tubo? 
	\item Repita lo anterior para un tubo de extremos abiertos.
\end{enumerate}


