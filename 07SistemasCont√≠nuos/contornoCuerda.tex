\subsection*{Condiciones de contorno para una Cuerda}

\item Se tiene una cuerda de longitud $L$ y densidad lineal de masa $\mu$ sometida a una tensión $T_{0}$.
Proponga como solución de la ecuación de ondas para un modo normal a la expresión: $\Psi(x,t)= A \sen{ \left(k x+ \varphi \right ) } \cos{ \left( \omega t + \theta \right) }$.
Tome el sistema de coordenadas con $x=0$ en un extremo de la cuerda y $x=L$ en el otro.
Encuentre la forma particular que adopta la solución propuesta en los siguientes casos: 
\begin{enumerate}
	\item $\Psi(0,t) = \Psi(L,t) = 0$ (ambos extremos están fijos). 
	\item $\Psi(0,t) = 0$ y $\pdv{\Psi}{x}$ $(L,t)= 0$ (un extremo está fijo y el otro está libre).
	% \item $\Psi(0,t)= 0$ y $\frac{\partial \Psi}{\partial x}(L,t)= 0$ (un extremo está fijo y el otro está libre).
	¿Imponer que un extremo se encuentre ``libre'' es equivalente a no imponer condiciones de contorno sobre ese extremo?
	¿Cómo lograría un extremo ``libre'' para la cuerda? 
	\item $\pdv{\Psi}{x}$$(0,t)=$$ \pdv{\Psi}{x}$ $(L,t)= 0$ (ambos extremos se encuentran libres).
	% \item $\frac{\partial \Psi}{\partial x}(0,t)= \frac{\partial\Psi}{\partial x}(L,t)=0$ (ambos extremos se encuentran libres).
	¿A qué corresponde el modo de frecuencia mínima?
	¿Cuál es la frecuencia de oscilación de ese modo? 
	\item Ahora tome un sistema de coordenadas con $x=0$ en el centro de la cuerda.
	Halle la forma que adopta la solución general propuesta si $\Psi(-L/2,t)= \Psi(L/2,t)= 0$ (ambos extremos fijos).
\end{enumerate}



\item Consideremos que las cuatro cuerdas de un violín son de igual longitud, y que emiten en su modo fundamental las notas: sol$_\text{2}= \SI{196}{\hertz}$, re$_\text{3}= \SI{294}{\hertz}$, la$_\text{3}= \SI{440}{\hertz}$ y mi$_\text{4}= \SI{659}{\hertz}$.
De la primera a la cuarta las cuerdas son de distinto material y diámetro:
\begin{tasks}[style=enumerate](2)
	\task \isotope{Al} \(\rho= \SI{2.6}{\gram\over\centi\metre\cubed}\), \(d_1= \SI{0,09}{\centi\metre}\)
	\task Aleación \isotope{Al}-\isotope{Ni} $\rho = \SI{1.2}{\gram\over\centi\metre\cubed}$, $d_2 = \SI{0.12}{\centi\metre}$
	\task Aleación \isotope{Al}-\isotope{Ni} $\rho = \SI{1.2}{\gram\over\centi\metre\cubed}$, $d_3 = \SI{0.1}{\centi\metre}$
	\task Acero $\rho = \SI{7.5}{\gram\over\centi\metre\cubed}$, $d_4 = \SI{0.1}{\centi\metre}$.
\end{tasks}
Calcular las tensiones a las que deben estar sometidas con respecto a la de la$_\text{3}$.



\item Se tiene una cuerda de \SI{20}{\centi\metre} de longitud y \SI{5}{\gram} de masa, sometida a una tensión de \SI{120}{\newton}.
Calcule sus modos naturales de oscilación.
¿Son todos audibles para el oído humano (\(\SI{20}{\hertz} < \nu < \SI{20}{\kilo\hertz}\))?



\item Una cuerda de longitud $L$ fija en sus extremos es lanzada a oscilar con igual amplitud en sus dos modos de menor frecuencia.
Considere que parte del reposo. 
\begin{enumerate}
	\item Encuentre el apartamiento del equilibrio para cada punto de la cuerda en función del tiempo.
	\item ¿Con qué período se repite el movimiento?
	\item Grafíquelo para cuatro instantes equiespaciados dentro de un período. 
\end{enumerate}


