\subsection*{Descomposición espectral de Condiciones iniciales}

\item
\begin{minipage}[t][2.3cm]{0.6\textwidth}
Una cuerda de densidad lineal de masa $\mu_{0}$ está sujeta en un extremo mientras el otro oscila libre manteniendo una tensión $T_{0}$.
En $t = 0$ se le impone la deformación dibujada (obvié el hecho de que eso es físicamente imposible sin modificar la homogeneidad de $\mu$).
La velocidad de propagación es \(v = \SI{80}{\metre\over\second} \).
\end{minipage}
\begin{minipage}[c][0.4cm][t]{0.34\textwidth}
	\begin{tikzpicture}
		\coordinate (quiebreInferior) at (1,0);
		\coordinate (quiebreSuperior) at (1,1);
		\coordinate (finalInferior) at (4,0);
		\coordinate (finalSuperior) at (4,1);
		\draw [-Latex] (0,0) -- (5,0) node [anchor=north] {\( x \)};	% eje x
		\draw [Latex-](0,2) node [anchor=west] {\( \psi(x,0) \)} -- (0,0) node [anchor=north] {\(0\)};	% eje y
		\draw [ultra thick] (0,0) -- (quiebreInferior);	% cuerda
		\draw [ultra thick] (quiebreSuperior) -- (finalSuperior);	% cuerda
		\draw [thin, dashed] (0,1) node [anchor=east] {\( \psi_0 \)} -- (quiebreSuperior);
		\draw [thin, dashed] (quiebreInferior) node [anchor=north] {\( L/4 \)} -- (quiebreSuperior);
		\draw [thin, dashed] (finalInferior) node [anchor=north] {\( L \)} -- (finalSuperior);
	\end{tikzpicture}
%	\includegraphics[width=\textwidth]{ej1-25}
\end{minipage}
\begin{enumerate}
	\item Halle $\psi(x,t)$ y grafíquelo para $\omega_1 t = 0,\,\pi$ y $2\pi$.
	\item Si tomara un sistema de coordenadas con el origen en el extremo libre de la cuerda, diga qué es lo que cambiaría.
	¿Es conveniente tal sistema?
\end{enumerate}





\item
\begin{minipage}[t][1.1cm]{0.6\textwidth}
¿Cuál \(L_1\) se maximiza la excitación del segundo modo?
¿Qué cambia musicalmente al cambiar \(L_1\)?
\end{minipage}
\begin{minipage}[c][1.5cm][t]{0.3\textwidth}
	\begin{tikzpicture}
		\draw [-Latex] (0,0) -- (5,0) node [anchor=north] {\( x \)};	% eje x
		\draw [Latex-](0,1) node [anchor=west] {\( \psi(x,0) \)} -- (0,0) node [anchor=north] {\(0\)};	% eje y
		\coordinate (L1) at (3,0.5);
		\coordinate (L1Inferior) at (3,0);
		\coordinate (L) at (4,0);
		\draw [ultra thick] (0,0) -- (L1) -- (L);	% cuerda
		\draw [thin, dashed] (0,0.5) node [anchor=east] {\( \psi_0 \)} -- (L1);
		\draw [thin, dashed] (L1Inferior) node [anchor=north] {\( L_1 \)} -- (L1);
		\draw [thin, dashed] (L) node [anchor=north] {\( L \)};
	\end{tikzpicture}
\end{minipage}



\item (*) Dada una cuerda de longitud $L$ y densidad de masa uniforme $\mu$, sometida a una tensión $T_{0}$ con ambos extremos fijos, demostrar que si $\psi(x,0)$ y $\dot{\psi}(x,0)$ son simétricas con respecto al centro de la cuerda, los modos con números de onda $k_{p}=2p\pi/L$ no se excitan.



\item
\begin{minipage}[t][2cm]{0.6\textwidth}
Un extremo de una cuerda de densidad lineal $\mu$ está fijo en tanto que está libre el que está a una distancia $L$.
Siempre se manteniendo una tensión $T_0$, en $t=0$ se la golpea sin deformarla pero imprimiéndole una velocidad $\dot{\psi}(x,0)$.
Halle $\psi(x,t>0)$.
\end{minipage}
\begin{minipage}[c][1cm][t]{0.49\textwidth}
	\begin{tikzpicture}
		\draw [-Latex] (0,0) -- (5,0) node [anchor=north] {\( x \)};	% eje x
		\draw [Latex-](0,1.5) node [anchor=west] {\( \dot{\psi}(x,0) \)} -- (0,0) node [anchor=north] {\(0\)};	% eje y
		\draw [ultra thick] (0,0) -- (2,0.5) -- (4,0.5);	% cuerda
		\draw [thin, dashed] (0,0.5) node [anchor=east] {\( \dot{\psi}_0 \)} -- (2,0.5);
		\draw [thin, dashed] (2,0) node [anchor=north] {\( L/2 \)} -- (2,0.5);
		\draw [thin, dashed] (4,0) node [anchor=north] {\( L \)} -- (4,0.5);
	\end{tikzpicture}
\end{minipage}


