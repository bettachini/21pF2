\subsection*{Evolución temporal de condiciones iniciales}

\item Los extremos fijos de una cuerda de longitud $L$ y densidad lineal $\mu_0$ la someten a una tensión $T_0$.
\begin{enumerate}
	\item Escriba la expresión más general posible para un modo normal en dicha cuerda y diga cuál es la velocidad de propagación de las ondas en ella.
	\item Determine con las condiciones de contorno los números de onda $k_p$, frecuencias y fases.
	Con esto, escriba la expresión general para una perturbación arbitraria $\psi(x,t)$.
	\item Obtenga $\psi(x,t)$ para el caso que parte del reposo con
	$
	\psi(x,0) = \sen{\left( \frac{3\pi x}{2L} \right)} \cos( \frac{\pi x}{2L} ).
	$
\end{enumerate}


\item Una cuerda de longitud $L$, densidad de masa uniforme $\mu_{0}$ está sujeta en ambos extremos lo que la somete a una tensión $T_{0}$.
A $t=0$ la cuerda se suelta de modo que su forma está dada por la siguiente función
$$
\psi(x,0) = \sen{ \left(\frac{\pi x}{L} \right) } + \frac{1}{3} \sen{ \left( \frac{3\pi x}{L} \right) } + \frac{1}{5} \sen{ \left( \frac{5\pi x}{L} \right) },
$$
si se toma un sistema de coordenadas tiene $x=0$ en un extremo de la soga y $x = L$ en el otro. 
%\begin{enumerate}
%	\item Halle $\psi(x,t)$.
%	\item 
	Si notamos la frecuencia fundamental como $\omega_{1}$, grafique $\psi(x,t)$ en $\omega_1 t = 0,\,\pi/5,\,\pi/3$ y $\pi/2$.
	¿Qué simetría tiene $\psi(x,t)$ en torno a $\omega_1 t = \pi/2$?
	¿Y de $\pi$?.
	¿Cómo sería $\psi(x,t)$ para $\omega_1 t = 2 \pi$?
%\end{enumerate}


\item Para una cuerda de longitud $L$, densidad lineal $\mu_{0}$ sometida a una tensión $T_0$ notamos su elongación transversal como $\psi(x,t)$.
\begin{enumerate}
	\item Escriba la expresión más general que representa un modo normal en
dicha cuerda, es decir, la expresión más general de una onda estacionaria.
	\item Sabiendo que la cuerda tiene un extremo libre y otro fijo, y que el
sistema de coordenadas con el que trabaja es tal que el extremo libre
está en $x=0$ y el extremo fijo está en $x=L$, imponga las condiciones
de contorno y determine las constantes pertinentes.
	\item Usando la relación de dispersión, obtenga las posibles frecuencias
temporales $\nu_{n}$. 
	\item Si $\psi(x,0)=0$ y $\dot{\psi}(x,0)=V_{0}\cos\left(\frac{3\pi}{2L}x\right)$, obtenga amplitud y fase de cada modo y luego $\psi(x,t)$.
\end{enumerate}


\item (*) Una cuerda de longitud $L$ sujeta en ambos extremos y sometida a una tensión $T_{0}$ consta de dos tramos de longitudes $L_1$ y $L_2$ y densidades de masa uniformes $\mu_1$ y $\mu_2$.
\begin{enumerate}
	\item Halle la expresión más general para un modo normal en dicha cuerda.
	Plantee las condiciones de contorno y halle las condiciones que deben cumplir los distintos parámetros.
	\item Halle los modos normales en este caso que $L_{1}=3L_{2}$ y $\mu_{2}=9\mu_{1}$. 
\end{enumerate}


\item (*) Una cuerda de densidad de masa uniforme $\mu$ y longitud $L$ está tensada $T_0$ entre extremos fijos.
Actúa una fuerza de amortiguamiento proporcional a su velocidad de oscilación.
Hallar la forma más general de $\psi(x,t)$.


