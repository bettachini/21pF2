\section*{Red por reflexión}

\item (*) Se desea estudiar la estructura de una banda en la proximidad de 4300 Å, utilizando una red plana de reflexión de 10 cm y 1200 líneas/mm.
Hallar:
\begin{enumerate}
\item El máximo orden observable. 
\item El mínimo ángulo de incidencia para el cual se observa. 
\item El mínimo intervalo de longitudes de onda resueltas. 
\item El orden intensificado. ¿Es ventajoso? Justifique su respuesta. 
\end{enumerate}


\item (*) Una red de fase por reflexión tiene 4800 facetas/cm y ha sido construida
para intensificar el primer orden, en $\lambda=0.6\,\mu$m. 
\begin{enumerate}
	\item Hallar el ángulo que forman las caras facetadas con el plano de la red. 
	\item Suponiendo incidencia normal, calcular la dispersión angular para esa $\lambda$. 
	\item Si se iluminase la red con $\lambda=0.48\,\mu$m, ¿qué órdenes se verían?
\end{enumerate}
