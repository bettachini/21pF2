\documentclass[11pt,spanish,a4paper]{article}
% Versión 1.er cuat 2021 Víctor Bettachini < bettachini@df.uba.ar >

% Versión 1.er cuat 2021 Víctor Bettachini < bettachini@df.uba.ar >

\usepackage[T1]{fontenc}
\usepackage[utf8]{inputenc}

\usepackage[spanish, es-tabla]{babel}
\def\spanishoptions{argentina} % Was macht dass?
% \usepackage{babelbib}
% \selectbiblanguage{spanish}
% \addto\shorthandsspanish{\spanishdeactivate{~<>}}

\usepackage{graphicx}
\graphicspath{{./figuras/}}
% \usepackage{float}

\usepackage[arrowdel]{physics}
\newcommand{\pvec}[1]{\vec{#1}\mkern2mu\vphantom{#1}}
% \usepackage{units}
\usepackage[separate-uncertainty=true, multi-part-units=single, locale=FR]{siunitx}
\usepackage{isotope} % $\isotope[A][Z]{X}\to\isotope[A-4][Z-2]{Y}+\isotope[4][2]{\alpha}

\usepackage{tasks}
\usepackage[inline]{enumitem}
% \usepackage{enumerate}

\usepackage{hyperref}

% \usepackage{amsmath}
% \usepackage{amstext}
\usepackage{amssymb}

\usepackage{tikz}
\usepackage{tikz-dimline}
\usetikzlibrary{math}
\usetikzlibrary{arrows.meta}
% \usetikzlibrary{snakes}
% \usetikzlibrary{calc}
\usetikzlibrary{decorations.pathmorphing}
\usetikzlibrary{patterns}

\usepackage[hmargin=1cm,vmargin=1.6cm,nohead]{geometry}
% \voffset-3.5cm
% \hoffset-3cm
% \setlength{\textwidth}{17.5cm}
% \setlength{\textheight}{27cm}

\usepackage{lastpage}
\usepackage{fancyhdr}
\pagestyle{fancyplain}
\fancyhead{}
\fancyfoot{{\tiny \textcopyright DF, FCEyN, UBA}}
\fancyfoot[C]{ {\tiny Actualizado al \today} }
\fancyfoot[RO, LE]{Pág. \thepage/\pageref{LastPage}}
\renewcommand{\headrulewidth}{0pt}
\renewcommand{\footrulewidth}{0pt}


\begin{document}
\begin{center}
\textbf{Física 2} (Físicos) \hfill \textcopyright {\tt DF, FCEyN, UBA}\\
	\textsc{\LARGE Propagación en medios dispersivos | Transformada de Fourier}
\end{center}

Los ejercicios con (*) entrañan una dificultad adicional. Son para investigar después de resolver los demás.



\begin{enumerate}

								
\section*{Transformada de Fourier}

\item Se quiere investigar la relación entre el ancho de un paquete y el desfasaje de las frecuencias que lo componen.
Ayuda: \( \int\limits_{-\infty}^{\infty} \operatorname{e}^{-\alpha x^2} \dd{x} = \sqrt{\frac{\pi}{\alpha} } \), \( \int\limits_{-\infty}^{\infty} \operatorname{e}^{-(x+ a)^2} \dd{x} = \sqrt{\pi}\).
\begin{enumerate}
	\item Tome el siguiente pulso con un espectro Gaussiano de ancho $\Delta k$ centrado en $k_0$ (note que las frecuencias están en fase):
$$
F(k)=A \operatorname{e}^{-\frac{ ( k - k_0 )^2 }{ 4 \Delta k^2 } }.
$$
Calcule $f(x) = \mathcal{F}^{-1}[F(k)]$ y vea que tiene una envolvente Gaussiana que modula una portadora de frecuencia $k_{0}$.
Note que el pulso está centrado en $x=0$ y que se cumple la relación $\Delta x \Delta k = 1/2$ (el paquete Gaussiano es el de mínima incerteza).
	\item Ahora desfase las distintas frecuencias en forma lineal, tal que:
$$
F(k)=A \operatorname{e}^{ -\frac{ ( k - k_0 )^2 }{ 4 \Delta k^2 } } \operatorname{e}^{ i \alpha (k - k_0 ) }.
$$
Calcule $f(x)$ y vea que es el mismo pulso que en la parte a), pero desplazado en $\alpha$ hacia la derecha (una fase lineal sólo corre la función).
	\item Ahora agregue una fase cuadrática, es decir:
$$
F(k) = A \operatorname{e}^{-\frac{(k-k_{0})^{2}}{4\Delta k^{2}} } \operatorname{e}^{i \beta ( k - k_0 )^2 }.
$$
Calcule $f(x)$ y vea que es un pulso Gaussiano centrado en $x=0$ pero con un ancho $\Delta x$ que cumple:
$$
\Delta x \Delta k = \frac{1}{2} \sqrt{ 1 + 16 \beta^2 \Delta k^4 }.
$$
¿Es cierto que si se quiere disminuir el ancho de un paquete siempre se debe aumentar $\Delta k$?
Derive $\Delta x$ con respecto a $\Delta k$ de la expresión anterior y analice lo pedido.
\end{enumerate}


\item (*) Muestre que si $\phi(t) \in \mathcal{R}$ y $\psi(\omega)= \mathcal{F} \left[ \phi (t) \right]$ es su transformada de Fourier, esta última cumple que \( \overline{\psi}(\omega) = \psi(- \omega) \), es decir, que para obtener su conjugada basta con invertir el signo de $\omega$.
	Aproveche esto para escribir a $\phi(t)$ como superposición de senos y cosenos.
	% \item Muestre que su transformada de Fourier $\psi(\omega)$ cumple $\psi(\omega)=\psi(-\omega)$.
	% cumple $\psi(\omega) = \psi^*(-\omega)$ ($\psi(\omega) = |\psi(-\omega)|$).



\subsection*{Propagación en medios dispersivos}

\item Se tiene un pulso de ancho $\Delta k$ centrado en $k_0$ tal que la siguiente es una buena aproximación para la relación de dispersión:
\[
	\omega(k) = \omega_0 (k_0) + \omega'(k_0) (k - k_0) + \frac{1}{2} \omega'' (k_0) ( k - k_0 )^2,
\]
donde \(\omega'= \pdv{\omega}{k}\) y  \(\omega''= \pdv[2]{\omega}{k}\).
Si en $t=0$ el pulso se propaga hacia $x<0$, y se escribe
\[
	\psi(x,0) = A \int_{-\infty}^{+\infty} \operatorname{e}^{ - \frac{ ( k - k_0 )^2 }{ 4 \Delta k^2 } } \operatorname{e}^{ i k x } \dd{k} + \qq{c.c.}.
\]
Calcule $\psi(x,t)$.
Vea cuál es la posición y el ancho del paquete como función del tiempo.
¿Es cierto que al viajar por un medio dispersivo cualquier paquete se ensancha?





\end{enumerate}

\end{document}
