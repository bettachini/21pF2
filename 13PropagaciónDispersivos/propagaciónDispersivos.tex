\section*{Propagación en medios dispersivos}

\item Se tiene un pulso de ancho $\Delta k$ centrado en $k_0$ tal que la siguiente es una buena aproximación para la relación de dispersión:
\[
	\omega(k) = \omega_0 (k_0) + \omega'(k_0) (k - k_0) + \frac{1}{2} \omega'' (k_0) ( k - k_0 )^2,
\]
donde \(\omega'= \pdv{\omega}{k}\) y  \(\omega''= \pdv[2]{\omega}{k}\).
Si en $t=0$ el pulso se propaga hacia $x<0$, y se escribe
\[
	\psi(x,0) = A \int_{-\infty}^{+\infty} \operatorname{e}^{ - \frac{ ( k - k_0 )^2 }{ 4 \Delta k^2 } } \operatorname{e}^{ i k x } \dd{k} + \qq{c.c.}.
\]
Calcule $\psi(x,t)$.
Vea cuál es la posición y el ancho del paquete como función del tiempo.
¿Es cierto que al viajar por un medio dispersivo cualquier paquete se ensancha?
