\section*{Propagación en medios}


\item 
\textbf{Interfaz entre medios no dispersivos}\\
\begin{minipage}[t][2.1cm]{0.6\textwidth}
Dos cuerdas semi-infinitas de distinta densidad lineal de masa, $\lambda_{m\,\text{izq}}$ y $\lambda_{m\,\text{der}}$, están unidas en un punto y sometidas a una tensión $T_0$.
Sobre la primera se propaga hacia la derecha el pulso que muestra la figura.
Se conocen $\lambda_{m\,\text{izq}}$, $\lambda_{m\,\text{der}}$, $T_0$, $\Delta x$ y $h$, y se considera que los medios son no dispersivos.
\end{minipage}
\begin{minipage}[c][0.4cm][t]{0.34\textwidth}
	\begin{tikzpicture}
%		\tikzmath{
%			\Deltax = 1;
%			\x0 = -1;
%			\haute = 1;
%			\xmin = -2.5;
%			\xmax = -\xmin;
%			\ymin = 0;
%			\ymax = 2;
%		}
		\def \Deltax{1};
		\def \x0{-1};
		\def \haute{1};
		\def \xmin{-3};
		\def \xmax{2.5};
		\def \ymin{0};
		\def \ymax{2};
		\coordinate (quiebreIzq) at ({(\x0-\Deltax/2)},0);
		\coordinate (quiebreSuperior) at ({\x0},\haute);
		\coordinate (quiebreDer) at ({(\x0+\Deltax/2)},0);
		\draw [-Latex, thin] (\xmin,0) -- (\xmax+0.3,0) node [anchor=north] {\(x\) [m]};	% eje x
		\draw [Latex-, thin](0,\ymax) node [anchor=west] {\( \psi(x,0) \)} -- (0,\ymin) node [anchor=north] {\(0\)};	% eje y
		\draw [ultra thick] (\xmin+0.3,0) -- (quiebreIzq) node [near start, above] {\(\lambda_{m\,\mathrm{izq}}\)} -- (quiebreSuperior) -- (quiebreDer) -- (0,0) ; %	% cuerda izq
		% \draw [ultra thick] (\xmin+0.3,0) -- (quiebreIzq) -- (quiebreSuperior) -- (quiebreDer) -- (0,0) node [near start, above] {\(\lambda_{m\,\mathrm{izq}}\)}; %	% cuerda izq
		\draw [ultra thick, gray] (0,0) -- ({\xmax-.3},0) node [near end, above] {\(\lambda_{m\,\mathrm{der}}\)};	% cuerda der
		\draw [thin, dashed] (0,\haute) node [anchor = west] {\( h \)} -- (quiebreSuperior);
		\dimline [label style= {above=0}, extension start length=-0.6, extension end length=-0.6]{(-1.5,-.6)}{(-0.5,-.6)}{\( \Delta x \)};
	\end{tikzpicture}
% 	\includegraphics[width=\textwidth]{ej2-20}
\end{minipage}
\begin{enumerate}
	\item Hallar el desplazamiento $\psi(x,t)$.
	\item Explique cualitativamente cómo cambian estos resultados si el medio es dispersivo.
\end{enumerate}



\item 
\textbf{Pulso cuadrados en medio dispersivo}\\
Se tiene un pulso de ancho $\Delta k$ centrado en $k_0$ tal que la siguiente es una buena aproximación para la relación de dispersión:
\[
	\omega(k) = \omega_0 (k_0) + \omega'(k_0) (k - k_0) + \frac{1}{2} \omega'' (k_0) ( k - k_0 )^2,
\]
donde \(\omega'= \pdv{\omega}{k}\) y  \(\omega''= \pdv[2]{\omega}{k}\).
Si en $t=0$ el pulso se propaga hacia $x<0$, y se escribe
\[
	\psi(x,0) = A \int_{-\infty}^{+\infty} \operatorname{e}^{ - \frac{ ( k - k_0 )^2 }{ 4 \Delta k^2 } } \operatorname{e}^{ i k x } \dd{k} + \qq{c.c.}.
\]
Calcule $\psi(x,t)$.
Vea cuál es la posición y el ancho del paquete como función del tiempo.
¿Es cierto que al viajar por un medio dispersivo cualquier paquete se ensancha?
