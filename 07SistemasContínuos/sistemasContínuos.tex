\documentclass[11pt,spanish,a4paper]{article}
% Versión 1.er cuat 2021 Víctor Bettachini < bettachini@df.uba.ar >

% Versión 1.er cuat 2021 Víctor Bettachini < bettachini@df.uba.ar >

\usepackage[T1]{fontenc}
\usepackage[utf8]{inputenc}

\usepackage[spanish, es-tabla]{babel}
\def\spanishoptions{argentina} % Was macht dass?
% \usepackage{babelbib}
% \selectbiblanguage{spanish}
% \addto\shorthandsspanish{\spanishdeactivate{~<>}}

\usepackage{graphicx}
\graphicspath{{./figuras/}{../LaTeX/}}
% \usepackage{float}

\usepackage[arrowdel]{physics}
\newcommand{\pvec}[1]{\vec{#1}\mkern2mu\vphantom{#1}}
% \usepackage{units}
\usepackage[separate-uncertainty=true, multi-part-units=single, locale=FR]{siunitx}
\usepackage{isotope} % $\isotope[A][Z]{X}\to\isotope[A-4][Z-2]{Y}+\isotope[4][2]{\alpha}

\usepackage{tasks}
\usepackage[inline]{enumitem}
% \usepackage{enumerate}

\usepackage{hyperref}

% \usepackage{amsmath}
% \usepackage{amstext}
\usepackage{amssymb}

\usepackage{tikz}
\usepackage{tikz-dimline}
\usetikzlibrary{calc}
% \usetikzlibrary{math}
\usetikzlibrary{arrows.meta}
\usetikzlibrary{snakes}
\usetikzlibrary{decorations}
\usetikzlibrary{decorations.pathmorphing}
\usetikzlibrary{patterns}

% \usepackage[hmargin=1cm, vmargin=1cm, includeheadfoot]{geometry}
\usepackage[hmargin=1cm,vmargin=3cm, top= 0.75cm,nohead]{geometry}
% \voffset-3.5cm
% \hoffset-3cm
% \setlength{\textwidth}{17.5cm}
% \setlength{\textheight}{27cm}

\usepackage{lastpage}
\usepackage{fancyhdr}
\pagestyle{fancyplain}
\fancyhf{}
% \fancyhead{}
\setlength\headheight{28.7pt} 
\fancyhead[LE, LO]{\textbf{Física 2} (Físicos) }
% \lhead{\textbf{Física 2} (Físicos) }
\fancyhead[RE, RO]{\href{https://df.uba.ar/es/}{$\vcenter{\hbox{\includegraphics[height=1cm]{sin_texto.pdf}}}$}}
% \rhead{$\vcenter{\hbox{\includegraphics[height=1cm]{sin_texto.jpg}}}$}
% \rhead{\includegraphics[height=1cm]{sin_texto.jpg}}
% \rhead{\textcopyright {\tt DF, FCEyN, UBA}}
\fancyfoot{\href{https://creativecommons.org/licenses/by-sa/4.0/deed.es/}{$\vcenter{\hbox{\includegraphics[height=0.4cm]{cc-by-sa.pdf}}}$} \href{https://df.uba.ar/es/}{DF, FCEyN, UBA}}
% \fancyfoot{$\vcenter{\hbox{\includegraphics[height=0.4cm]{cc-by-sa.pdf}}}$ DF, FCEyN, UBA}
% \fancyfoot{{\tiny \textcopyright DF, FCEyN, UBA}}
\fancyfoot[C]{ {\tiny Actualizado al \today} }
\fancyfoot[RO, LE]{Pág. \thepage/\pageref{LastPage}}
\renewcommand{\headrulewidth}{0pt}
\renewcommand{\footrulewidth}{0pt}


\begin{document}
\begin{center}
\textbf{Física 2} (Físicos) \hfill \textcopyright {\tt DF, FCEyN, UBA}\\
	\textsc{\LARGE Sistemas contínuos}
\end{center}

Los ejercicios con (*) son opcionales.

\begin{enumerate}


\subsection*{Cuerda}

\item Se tiene una cuerda de longitud $L$ y densidad lineal de masa $\mu$ sometida a una tensión $T_{0}$.
Proponga como solución de la ecuación de ondas para un modo normal a la expresión: $\Psi(x,t)= A \sen{ \left(k x+ \varphi \right ) } \cos{ \left( \omega t + \theta \right) }$.
Tome el sistema de coordenadas con $x=0$ en un extremo de la cuerda y $x=L$ en el otro.
Encuentre la forma particular que adopta la solución propuesta en los siguientes casos: 
\begin{enumerate}
	\item $\Psi(0,t) = \Psi(L,t) = 0$ (ambos extremos están fijos). 
	\item $\Psi(0,t) = 0$ y $\pdv{\Psi}{x}$ $(L,t)= 0$ (un extremo está fijo y el otro está libre).
	% \item $\Psi(0,t)= 0$ y $\frac{\partial \Psi}{\partial x}(L,t)= 0$ (un extremo está fijo y el otro está libre).
	¿Imponer que un extremo se encuentre ``libre'' es equivalente a no imponer condiciones de contorno sobre ese extremo?
	¿Cómo lograría un extremo ``libre'' para la cuerda? 
	\item $\pdv{\Psi}{x}$$(0,t)=$$ \pdv{\Psi}{x}$ $(L,t)= 0$ (ambos extremos se encuentran libres).
	% \item $\frac{\partial \Psi}{\partial x}(0,t)= \frac{\partial\Psi}{\partial x}(L,t)=0$ (ambos extremos se encuentran libres).
	¿A qué corresponde el modo de frecuencia mínima?
	¿Cuál es la frecuencia de oscilación de ese modo? 
	\item Ahora tome un sistema de coordenadas con $x=0$ en el centro de la cuerda.
	Halle la forma que adopta la solución general propuesta si $\Psi(-L/2,t)= \Psi(L/2,t)= 0$ (ambos extremos fijos).
\end{enumerate}


\item Se tiene una cuerda de \SI{20}{\centi\metre} de longitud y \SI{5}{\gram} de masa, sometida a una tensión de \SI{120}{\newton}.
Calcule sus modos naturales de oscilación.
¿Son todos audibles para el oído humano (\(\SI{20}{\hertz} < \nu < \SI{20}{\kilo\hertz}\))?


\item Consideremos que las cuatro cuerdas de un violín son de igual longitud, y que emiten en su modo fundamental las notas: sol$_\text{2}= \SI{196}{\hertz}$, re$_\text{3}= \SI{294}{\hertz}$, la$_\text{3}= \SI{440}{\hertz}$ y mi$_\text{4}= \SI{659}{\hertz}$.
De la primera a la cuarta las cuerdas son de distinto material y diámetro:
\begin{tasks}[style=enumerate](2)
	\task \isotope{Al} \(\rho= \SI{2.6}{\gram\over\centi\metre\cubed}\), \(d_1= \SI{0,09}{\centi\metre}\)
	\task Aleación \isotope{Al}-\isotope{Ni} $\rho = \SI{1.2}{\gram\over\centi\metre\cubed}$, $d_2 = \SI{0.12}{\centi\metre}$
	\task Aleación \isotope{Al}-\isotope{Ni} $\rho = \SI{1.2}{\gram\over\centi\metre\cubed}$, $d_3 = \SI{0.1}{\centi\metre}$
	\task Acero $\rho = \SI{7.5}{\gram\over\centi\metre\cubed}$, $d_4 = \SI{0.1}{\centi\metre}$.
\end{tasks}
Calcular las tensiones a las que deben estar sometidas con respecto a la de la$_\text{3}$.


\item Una cuerda de longitud $L$ fija en sus extremos es lanzada a oscilar con igual amplitud en sus dos modos de menor frecuencia.
Considere que parte del reposo. 
\begin{enumerate}
	\item Encuentre el apartamiento del equilibrio para cada punto de la cuerda en función del tiempo.
	\item ¿Con qué período se repite el movimiento?
	\item Grafíquelo para cuatro instantes equiespaciados dentro de un período. 
\end{enumerate}


\subsection*{Gas en un tubo unidimensional}

\item Se tiene un tubo lleno de aire de longitud $L$. Considere las siguientes posibilidades: 
\begin{tasks}(3)
	\task Cerrado en ambos extremos.
	\task Uno abierto el otro cerrado.
	\task Ambos extremos abiertos.
\end{tasks}
Asuma conocidos: $L$, $v_\text{sonido}$, la presión atmosférica $p_0$, $\rho_{0}= \frac{ \gamma p_0 }{ v_\text{sonido}^2}$ y que $\gamma= \frac{C_p}{C_v}= \frac{7}{5}$ para un gas diatómico.
Halle para cada una de dichas situaciones: 
\begin{enumerate}
	\item Las posibles longitudes de onda con las que puede vibrar el aire en el tubo, y sus correspondientes frecuencias. 
	\item Elija un sistema de referencia conveniente, y escriba la expresión más general para el desplazamiento de las partículas $\Psi(x,t)$.
	¿Qué parámetros conoce de dicha expresión?
	¿De qué dependen los que no conoce? 
	\item A partir de la expresión de $\Psi(x,t)$, hallar el apartamiento de la presión respecto a la atmosférica $\delta p(x,t)$.
	¿Cuál es la diferencia de fase entre ellas?
	¿Cuál es la amplitud máxima de presión? 
	\item Hallar la función de densidad $\rho(x,t)$.
	¿Cuál es amplitud máxima?
\end{enumerate}


\item 
\begin{enumerate}
\item ¿Qué longitud debe tener un tubo de órgano abierto en ambos extremos para producir un sonido de \SI{440}{\hertz}? 
\item Si uno de sus extremos está cerrado y se desea producir el mismo tono en su primer armónico, ¿qué longitud deberá tener?
\end{enumerate}


\item Se tiene un tubo cerrado en uno de sus extremos; su longitud es menor a \SI{1}{\metre}.
Se acerca al extremo abierto un diapasón que está vibrando con $\nu = \SI{440}{\hertz}$.
Considere $v_\text{sonido} = \SI{330}{\metre\over\second}$.
\begin{enumerate}
	\item Hallar las posibles longitudes del tubo para que haya resonancia.
	Para cada una de ellas, ¿en qué modo está vibrando el aire contenido en el tubo? 
	\item Repita lo anterior para un tubo abierto en ambos extremos.
\end{enumerate}


\end{enumerate}

\end{document}
