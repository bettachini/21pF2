\section*{Propagación en medios no dispersivos}


\item \label{propagando}
\begin{minipage}[t][2.6cm]{0.4\textwidth}
\textbf{Perturbacion propagando en una cuerda no dispersiva}\\
% \fussy
Una perturbación se propaga en una cuerda infinita con velocidad $v$.
Las figuras la muestran en $t =0$ y $t = \SI{4}{\second}$.
Determine $v$ y $\psi(x,t)$.
\end{minipage}
\begin{minipage}[c][1.1cm][t]{0.54\textwidth}
	\begin{tikzpicture}
%		\tikzmath{
%			\Dt= 2;
%			\t0= 1;
%			\haute= 1;
%			\xmin= -1;
%			\xmax= 6.5;
%			\ymin= 0;
%			\ymax= 2;
%		}
		\def \Dt{2};
		\def \t0{1};
		\def \haute{1};
		\def \xmin{-1};
		\def \xmax{6.5};
		\def \ymin{0};
		\def\ymax{2};
		\coordinate (quiebreInferior) at (\t0/2,0);
		\coordinate (quiebreSuperior) at ({(\t0+\Dt)/2},\haute);
		\coordinate (quiebreSuperiorDebajo) at ({(\t0+\Dt)/2},0);
		\draw [-Latex] (\xmin/2,0) -- (\xmax/2,0) node [anchor=north] {\(x\) [m]};	% eje x
		\draw [Latex-](0,\ymax) node [anchor=west] {\( \psi(x,0) \) [cm]} -- (0,\ymin) node [anchor=north] {\(0\)};	% eje y
		\draw [ultra thick] (\xmin/2,0) -- (quiebreInferior) node [anchor = north] {\t0} -- (quiebreSuperior) -- (quiebreSuperiorDebajo) node [anchor = north] {\( 3 \)} -- (\xmax/2-0.15,0);	% cuerda
		% \draw [ultra thick] (\xmin/2,0) -- (quiebreInferior) node [anchor = north] {\t0} -- (quiebreSuperior) node [anchor = south west] {\(t=0\)}-- (quiebreSuperiorDebajo) node [anchor = north] {3} -- (\xmax/2-0.15,0);	% cuerda
		\draw [thin, dashed] (0,\haute) node [anchor=east] {\( 1 \)} -- (quiebreSuperior);
	\end{tikzpicture}
	\quad
	\begin{tikzpicture}
%		\tikzmath{
%			\Dt= 2;
%			\t0= 1;
%			\haute= 1;
%			\xmin= -1;
%			\xmax= 6.5;
%			\ymin= 0;
%			\ymax= 2;
%		}
		\def \Dt{2};
		\def \t0{1};
		\def \haute{1};
		\def \xmin{-1};
		\def \xmax{6.5};
		\def \ymin{0};
		\def \ymax{2};
		\coordinate (quiebreInferior) at ({(\t0+\Dt)/2},0);
		\coordinate (quiebreSuperior) at ({(\t0+\Dt+\Dt)/2},\haute);
		\coordinate (quiebreSuperiorDebajo) at ({(\t0+\Dt+\Dt)/2},0);
		\draw [-Latex] (\xmin/2,0) -- (\xmax/2,0) node [anchor=north] {\(x\) [m]};	% eje x
		\draw [Latex-](0,\ymax) node [anchor=west] {\( \psi(x,4 s) \) [cm]} -- (0,\ymin) node [anchor=north] {\(0\)};	% eje y
		\draw [ultra thick] (\xmin/2,0) -- (quiebreInferior) node [anchor = north] {\( 3 \)} -- (quiebreSuperior) -- (quiebreSuperiorDebajo) node [anchor = north] {\( 5 \)} -- (\xmax/2-0.15,0);	% cuerda
		\draw [thin, dashed] (0,\haute) node [anchor=east] {\( 1 \)} -- (quiebreSuperior);
	\end{tikzpicture}
\end{minipage}



\item 
\textbf{Perturbacion inicial en una cuerda no dispersiva}\\
Suponga que la cuerda del problema \ref{propagando} fue soltada desde el reposo con la deformación vista en $t=0$.
\begin{enumerate}
	\item Escriba las componentes de la perturbación que se propagan a izquierda y derecha que conforman $\psi(x,t) = \psi_\text{derecha} (x - v t ) + \psi_\text{izquierda} ( x + v t )$.
	\item Grafique $\psi(x, t= \SI{0.25}{s})$.
\end{enumerate}



\item (*) Ambos extremos de una cuerda de densidad $\mu$ están fijos sometiéndola a una tensión $T$.
A $t=0$ se la suelta con $h \ll L$ desde
$
\psi(x,0)=\begin{cases}
0 & \mbox{si }0<x<a\\
h\frac{x-a}{L/2-a} & \mbox{si }a<x<L/2\\
h\frac{L-a-x}{L/2-a} & \mbox{si }L/2<x<L-a\\
0 & \mbox{si }L-a<x<L .
\end{cases}
$
\begin{enumerate}
	\item Hallar $\psi(x,t)$ y demostrar que siempre es posible escribir esta solución como una superposición de una onda que se propaga hacia la derecha y una que se propaga hacia la izquierda.
	\item Hacer un esquema cualitativo del movimiento de la cuerda para los instantes $t_n = \frac{n}{8} \frac{L}{v}$, donde $v$ es la velocidad de propagación de las ondas en la cuerda y $n$ es un número natural.
\end{enumerate}


\item 
\begin{minipage}[t][1.5cm]{0.7\textwidth}
(*) En un gas, a $t=0$, se produce la perturbación sobre su densidad \(\rho\) indicada en la figura.
Conociendo la $v_\text{sonido}$, $\rho_{1}$, $\rho_{0}$ tales que $(\rho_{1}-\rho_{0})/\rho_{0}\ll1$ y que en ese momento el gas estaba en reposo, calcule $\rho(x,t)$.
\end{minipage}
\begin{minipage}[c][1.6cm][t]{0.2\textwidth}
	\begin{tikzpicture}
%		\tikzmath{
%			\rrho0= 0.3;
%			\rrho1= 0.8;
%			\xmax= 2;
%			\xmin= -\xmax;
%			\ymax= 1.5;
%			\ymin= 0;
%		}
		\def \rrhoCero{0.3};
		\def \rrhoUno{0.8};
		\def \xmax{2};
		\def \xmin{-\xmax};
		\def \ymax{1.5};
		\def \ymin{0};
		\draw [-Latex] (\xmin,0) -- (\xmax,0) node [anchor=north] {\( x \)};	% eje x
		\draw [Latex-](0,\ymax) node [anchor=west] {\( \rho(x,0) \)} -- (0,\ymin) node [anchor=north] {\(0\)};	% eje y
		\draw [ultra thick] (0,\rrhoCero) node [anchor = east] {\( \rho_0 \)} -- (\xmax,\rrhoCero);
		\draw [ultra thick] (0,\rrhoUno) node [anchor = west] {\( \rho_1 \)} -- (\xmin,\rrhoUno);
	\end{tikzpicture}
	% \includegraphics[width=\textwidth]{ej2-5}
\end{minipage}


\item 
\begin{minipage}[t][2.1cm]{0.6\textwidth}
Dos cuerdas semi-infinitas de distinta densidad lineal de masa, $\lambda_{m\,\text{izq}}$ y $\lambda_{m\,\text{der}}$, están unidas en un punto y sometidas a una tensión $T_0$.
Sobre la primera se propaga hacia la derecha la perturbación que muestra la figura.
Se conocen $\lambda_{m\,\text{izq}}$, $\lambda_{m\,\text{der}}$, $T_0$, $\Delta x$ y $h$, y se considera que los medios son no dispersivos.
\end{minipage}
\begin{minipage}[c][0.4cm][t]{0.34\textwidth}
	\begin{tikzpicture}
%		\tikzmath{
%			\Deltax = 1;
%			\x0 = -1;
%			\haute = 1;
%			\xmin = -2.5;
%			\xmax = -\xmin;
%			\ymin = 0;
%			\ymax = 2;
%		}
		\def \Deltax{1};
		\def \x0{-1};
		\def \haute{1};
		\def \xmin{-3};
		\def \xmax{2.5};
		\def \ymin{0};
		\def \ymax{2};
		\coordinate (quiebreIzq) at ({(\x0-\Deltax/2)},0);
		\coordinate (quiebreSuperior) at ({\x0},\haute);
		\coordinate (quiebreDer) at ({(\x0+\Deltax/2)},0);
		\draw [-Latex, thin] (\xmin,0) -- (\xmax+0.3,0) node [anchor=north] {\(x\) [m]};	% eje x
		\draw [Latex-, thin](0,\ymax) node [anchor=west] {\( \psi(x,0) \)} -- (0,\ymin) node [anchor=north] {\(0\)};	% eje y
		\draw [ultra thick] (\xmin+0.3,0) -- (quiebreIzq) node [near start, above] {\(\lambda_{m\,\mathrm{izq}}\)} -- (quiebreSuperior) -- (quiebreDer) -- (0,0) ; %	% cuerda izq
		% \draw [ultra thick] (\xmin+0.3,0) -- (quiebreIzq) -- (quiebreSuperior) -- (quiebreDer) -- (0,0) node [near start, above] {\(\lambda_{m\,\mathrm{izq}}\)}; %	% cuerda izq
		\draw [ultra thick, gray] (0,0) -- ({\xmax-.3},0) node [near end, above] {\(\lambda_{m\,\mathrm{der}}\)};	% cuerda der
		\draw [thin, dashed] (0,\haute) node [anchor = west] {\( h \)} -- (quiebreSuperior);
		\dimline [label style= {above=0}, extension start length=-0.6, extension end length=-0.6]{(-1.5,-.6)}{(-0.5,-.6)}{\( \Delta x \)};
	\end{tikzpicture}
% 	\includegraphics[width=\textwidth]{ej2-20}
\end{minipage}
\begin{enumerate}
	\item Hallar el desplazamiento $\psi(x,t)$.
	\item Explique cualitativamente cómo cambian estos resultados si el medio es dispersivo.
\end{enumerate}




\subsection*{Velocidad de fase y de grupo}

\item ¿Cuál de estos métodos determina la velocidad de fase y cuál la de grupo?
\begin{enumerate}
\item Golpear las manos y determinar el tiempo que transcurre entre el aplauso y el eco de un reflector ubicado a una distancia conocida.
\item Medir la longitud de un tubo que resuena a una frecuencia conocida (y corregir por efectos de borde).
\item Medir el tiempo en que el pulso de un láser recorre una distancia conocida.
\item Encontrar la longitud de una cavidad resonante que oscila a en modo y frecuencia conocidos.
\end{enumerate}


\item Demuestre que la relación de la velocidad de grupo $v_g$ y de fase $v_f$ es
$$
v_g = v_f - \lambda \dv{v_f}{\lambda}.
$$
¿Cómo es $\dv{v_f}{\lambda}$ en un medio no dispersivo?
En tal caso, ¿cuál es la relación entre $v_g$ y $v_f$?
