\section*{Polarizador}

\item Se hace incidir luz plano polarizada normalmente sobre un polarizador lineal.
Al ir rotando la lámina, ¿cómo varían el estado de polarización y la intensidad del haz transmitido? Indique a partir de qué dirección
mide el ángulo de giro.



\item Una onda inicialmente polarizada en \(x\) y viajando en \(z\) positivo incide en un polarizador cuyo eje de transmisión forma un ángulo \(\alpha\) con el eje \(x\), y luego pasa por un segundo polarizador que forma un ángulo \(\beta\) con el primero.
¿Cuál es la expresión de la onda transmitida en los ejes originales?
¿Y en los ejes del segundo polarizador?
¿Cuál es la intensidad media transmitida por el sistema?



\item Sobre un polarizador incide una onda circularmente polarizada en sentido horario.
¿Cuál es el estado de polarización de la onda transmitida?
¿Qué fracción de la intensidad incidente se transmitió a través de la lámina?



\item Incide un haz de luz natural de intensidad $I_0$ sobre un polarizador lineal (ideal). ¿Qué intensidad se transmite? ¿Por qué?



\item Sobre un polarizador lineal (ideal) incide una onda cuyo estado de polarización no se conoce, con una intensidad $I_0$.
Se hace girar esa lámina y se observa que la intensidad transmitida es $I_0/2$ y no depende del ángulo de giro.
¿Qué puede decir sobre el estado de polarización de la onda incidente?



\subsection*{Ley de Malus}

\item Dos polarizadores están orientados de manera que se transmita la máxima cantidad de luz.
¿A qué fracción de este valor máximo se reduce la intensidad de la luz transmitida cuando se gira el segundo polarizador en:
\begin{tasks}(3)
	\task \ang{20;;}
	\task \ang{45;;}
	\task \ang{60;;}
\end{tasks}



\item 
Se tienen dos polarizadores.
¿Cuál es el ángulo formado por sus ejes de transmisión si al incidir un haz de luz natural sobre el primero, un detector ubicado a la salida del segundo mide la cuarta parte de la energía emitida por la fuente?



\item Se tienen dos polarizadores cuyos ejes de transmisión forman un ángulo de \ang{45;;}.
Sobre el primero incide una onda circularmente polarizada en sentido horario.
¿Qué fracción de la intensidad incidente se transmitió a la salida del segundo polarizador?
