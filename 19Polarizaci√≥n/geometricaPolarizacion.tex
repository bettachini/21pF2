\section*{Descripción geométrica del estado de polarización de una onda}

\item ¿Cuándo dos ondas transversales, perpendiculares entre sí, dan una onda:
\begin{itemize}
	\item linealmente polarizada;
	\item circularmente polarizada en sentido antihorario;
	\item circularmente polarizada en sentido horario;
	\item elípticamente polarizada en sentido antihorario? 
\end{itemize}



\item Escriba la expresión matemática de:
\begin{enumerate}
	\item Una onda linealmente polarizada cuyo plano de polarización forma un ángulo de \ang{30;;} con el eje $x$ (se propaga según el eje $z$). 
	\item Una onda que se propaga según el eje $x$, polarizada circularmente en sentido horario.
	\item Una onda elípticamente polarizada en sentido horario, que se propaga según el eje $x$, tal que el eje mayor, que es igual a dos veces el eje menor, está sobre el eje $y$.
	\item Una onda elípticamente polarizada en sentido antihorario.
	La onda se propaga según el eje $x$ positivo (use terna directa). 
\end{enumerate}



\item Demuestre que siempre se puede describir una onda, cualquiera sea su polarización, como suma de dos ondas circularmente polarizadas en sentidos horario y antihorario.
