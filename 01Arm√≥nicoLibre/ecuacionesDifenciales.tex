\documentclass[11pt,spanish,a4paper]{article}
% Versión 1.er cuat 2021 Víctor Bettachini < bettachini@df.uba.ar >

% Versión 1.er cuat 2021 Víctor Bettachini < bettachini@df.uba.ar >

\usepackage[T1]{fontenc}
\usepackage[utf8]{inputenc}

\usepackage[spanish, es-tabla]{babel}
\def\spanishoptions{argentina} % Was macht dass?
% \usepackage{babelbib}
% \selectbiblanguage{spanish}
% \addto\shorthandsspanish{\spanishdeactivate{~<>}}

\usepackage{graphicx}
\graphicspath{{./figuras/}}
% \usepackage{float}

\usepackage[arrowdel]{physics}
\newcommand{\pvec}[1]{\vec{#1}\mkern2mu\vphantom{#1}}
% \usepackage{units}
\usepackage[separate-uncertainty=true, multi-part-units=single, locale=FR]{siunitx}
\usepackage{isotope} % $\isotope[A][Z]{X}\to\isotope[A-4][Z-2]{Y}+\isotope[4][2]{\alpha}

\usepackage{tasks}
\usepackage[inline]{enumitem}
% \usepackage{enumerate}

\usepackage{hyperref}

% \usepackage{amsmath}
% \usepackage{amstext}
\usepackage{amssymb}

\usepackage{tikz}
\usepackage{tikz-dimline}
\usetikzlibrary{math}
\usetikzlibrary{arrows.meta}
% \usetikzlibrary{snakes}
% \usetikzlibrary{calc}
\usetikzlibrary{decorations.pathmorphing}
\usetikzlibrary{patterns}

\usepackage[hmargin=1cm,vmargin=1.6cm,nohead]{geometry}
% \voffset-3.5cm
% \hoffset-3cm
% \setlength{\textwidth}{17.5cm}
% \setlength{\textheight}{27cm}

\usepackage{lastpage}
\usepackage{fancyhdr}
\pagestyle{fancyplain}
\fancyhead{}
\fancyfoot{{\tiny \textcopyright DF, FCEyN, UBA}}
\fancyfoot[C]{ {\tiny Actualizado al \today} }
\fancyfoot[RO, LE]{Pág. \thepage/\pageref{LastPage}}
\renewcommand{\headrulewidth}{0pt}
\renewcommand{\footrulewidth}{0pt}


\begin{document}
\begin{center}
\textbf{Física 2} (Físicos) \hfill \textcopyright {\tt DF, FCEyN, UBA}\\
	\textsc{\LARGE Repaso de ecuaciones diferenciales lineales}\\
\end{center}


Si puede terminar este breve ejercicio dispone del bagaje matemático necesario para comenzar la materia.

\begin{enumerate}

\item La siguiente es una ecuación lineal diferencial de segundo grado
\[
	\ddot{x} + a x = 0. \notag
\]
\begin{enumerate}
	\item	Halle la solución más general posible que será una combinación lineal de las soluciones que vaya hallando.
	Para hacer esto ensaye las siguientes soluciones
	\begin{enumerate}
		\item \( x(t) = A \cos(\omega t + \varphi ) \),
		\item \( x(t) = A_1 \cos( \omega t) + B_1 \sen(\omega t) \), y
		\item \( x(t) = C \operatorname{e}^{\lambda t} \) .
	\end{enumerate}
	\item ¿Qué representan \(\omega\) y \(\lambda\)?
	\item Verifique que las soluciones obtenidas son equivalentes entre sí.
	O dicho de otra manera, encuentre las relaciones entre \(A, A_1, B_1\) y \( C\).
	Para eso use las siguientes relaciones, que usaremos \textbf{muy frecuentemente} en la materia:
	\begin{itemize}
		\item \(\operatorname{e}^{i \alpha}= \cos(\alpha) + i \sen(\alpha) \), esta es la \textit{fórmula de Euler},
		\item \( \sen( \alpha \pm \beta ) = \sen(\alpha) \cos(\beta) \pm \cos(\alpha) \sen(\beta) \), y
		\item \( \cos( \alpha \pm \beta ) = \cos(\alpha) \cos(\beta) \mp \sen(\alpha) \sen(\beta) \).
	\end{itemize}
\end{enumerate}



\end{enumerate}

\end{document}
