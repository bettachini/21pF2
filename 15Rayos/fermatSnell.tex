\section*{Fermat | Ibn Sahl - Snell}

\item
A partir del principio de Fermat deducir la ley de Ibn Sahl - Snell para la refracción de la luz entre dos medios de índices $n_1$ y $n_2$, separados por una superficie plana.


\item
\begin{enumerate}
	\item (*) Si un rayo parte del punto $A=(0,1,0)$, se refleja en el espejo plano $(x,0,z)$ y pasa por el punto $B=(4,3,0)$, averigüe en qué punto sobre el plano del espejo se refleja y los ángulos de incidencia y reflexión.
	Aplicar Fermat e interpretar físicamente las soluciones. 
	\item (*) Un rayo directo entre $A$ y $B$ recorre un menor camino óptico que el hallado en (a), ¿es esto contradictorio?
\end{enumerate}



%\item (*) Un espejo elíptico de focos $A$ y $B$, tiene el primero una fuente puntual.
%Los espejos esférico y plano dibujados son tangentes al elíptico en $C$.
%Sabiendo que el camino óptico de un rayo que sale de $A$, se refleja en $C$ y luego pasa por $B$, es estacionario en la elipse, obtenga cualitativamente si el camino óptico es máximo, mínimo o estacionario cuando se refleja en cada uno de los espejos.
