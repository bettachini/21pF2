\section*{Fermat | Ibn Sahl - Snell}

\item
A partir del principio de Fermat deducir la ley de Ibn Sahl - Snell para la refracción de la luz entre dos medios de índices $n_1$ y $n_2$, separados por una superficie plana.


\item
\begin{enumerate}
	\item (*) Si un rayo parte del punto $A=(0,1,0)$, se refleja en el espejo plano $(x,0,z)$ y pasa por el punto $B=(4,3,0)$, averigüe en qué punto sobre el plano del espejo se refleja y los ángulos de incidencia y reflexión.
	Aplicar Fermat e interpretar físicamente las soluciones. 
	\item (*) Un rayo directo entre $A$ y $B$ recorre un menor camino óptico que el hallado en (a), ¿es esto contradictorio?
\end{enumerate}


\item \textbf{Índice de refracción}
\begin{enumerate}
	\item Un rayo de luz llega a la interfaz aire-líquido con un angulo de \ang{55}.
	Se observa que el rayo refractado se transmite a \ang{40}.
	¿Cuál es el índice de refracción del líquido?
	\item Un haz de luz incide desde el aire (\(n= 1\)) sobre una lámina de vidrio de índice de refracción desconocido y espesor \(d\).
	Al otro lado del vidrio hay agua de índice de refracción \num{1.33}.
	El ángulo de incidencia en la interfase aire-vidrio es \ang{30}.
	Calcule el ángulo que el rayo refractadoforma con la normal a la superficie en el agua.
	\item Un rayo de luz, que se propaga en un medio cuyo índice de refracción es \(n= 2\) incide formando un ángulo de \ang{30} respecto a la normal a la superficie de separación, que la separa de otro medio de índice \num{1.5}.
	Calcule el ángulo que forma el rayo transmitido con la normal a la superficie.
	Calcule el ángulo mínimo con el que debería incidir el rayo para que no se transmita nada.
\end{enumerate}



%\item (*) Un espejo elíptico de focos $A$ y $B$, tiene el primero una fuente puntual.
%Los espejos esférico y plano dibujados son tangentes al elíptico en $C$.
%Sabiendo que el camino óptico de un rayo que sale de $A$, se refleja en $C$ y luego pasa por $B$, es estacionario en la elipse, obtenga cualitativamente si el camino óptico es máximo, mínimo o estacionario cuando se refleja en cada uno de los espejos.
