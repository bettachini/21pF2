\section*{Limitación de la resolución por difracción}

\item Sean dos fuentes puntuales incoherentes colocadas en el plano focal objeto de una lente convergente; ambas emiten la misma $\lambda$.
A la derecha de la lente hay una ranura de ancho $b$, y luego una segunda lente.
Se observa la figura de difracción de Fraunhofer de las fuentes.
\begin{enumerate}
	\item Calcule la mínima separación angular entre las fuentes, y la correspondiente mínima separación lineal, para que las imágenes estén justamente resueltas según el criterio de Rayleigh.
	Discuta los casos en que ambas fuentes emiten con la misma irradiancia, y en que no. 
	\item Repita el cálculo efectuado en a) si la rendija se reemplaza por una abertura circular de diámetro $d$.
\end{enumerate}



\item Suponga al ojo humano limitado por difracción, y calcule el mínimo ángulo que resuelve para un diámetro de pupila de \SI{2}{\milli\metre}.
Si dos puntos se hallan a la distancia de visión clara, ¿cuál es la mínima distancia entre ellos para que estén justamente resueltos?
