\section*{Parámetros de una onda propagante}

\item Verifique si las siguientes expresiones matemáticas cumplen la ecuación
de las ondas unidimensional.
Grafique las funciones dadas.
\begin{tasks}(3)
	\task $\Psi(x,t)= A \operatorname{e}^{- \lambda ( x - v t)^2 }$
	\task $\Psi(x,t)= \beta ( x + v t )$
	\task $\Psi(x,t)= A \sen{\left[ k (x - v t) \right] }$
	\task $\Psi(x,t)= B \sen^2{ \left( k x -\omega t \right) }$
	\task $\Psi(x,t)= C \cos{(k x)} \sen{ \left( \omega t \right) }$
	\task $\Psi(x,t)= D \operatorname{e}^{i ( k x - \omega t ) }$
\end{tasks}


\item La ecuación de una onda transversal en una cuerda está dada por: $y(x,t) = \SI{0.1}{\metre} \sen{ \left( x \SI{\pi}{\per\metre} - t \SI{4 \pi}{\per\second} \right) }$.
Determine para la onda que se propaga en ella:
\begin{tasks}(2)
	\task amplitud,
	\task frecuencia de vibración, y
	\task velocidad de propagación.
	\task $x = \SI{2}{\metre}$ y $ t = \SI{1}{\second}$, desplazamiento, velocidad y la aceleración de la cuerda.
\end{tasks}


\item La frecuencia angular y número de onda de una onda transversal que se propaga en $\hat{x}$ es $\omega= \SI{10}{\per\second}$ y $k = \SI{100}{\per\metre}$.
En $x_1 = \SI{1}{\kilo\metre}$ y $t_1 = \SI{1}{\second}$ tiene por fase $\phi = \frac{3 \pi}{2}$.
\begin{tasks}
	\task ¿Cuál es la fase en ese mismo punto para $t = 0$?
	\task Considerando que $\phi(x,t) = k x - \omega t+ \phi_0$, ¿cuánto vale $\phi_0$?
	\task ¿A qué velocidad se propaga la onda?
	\task ¿En qué tiempo el frente de onda arriba a un $x_2 = 2 x_1$?
\end{tasks}


\item Una cuerda con densidad lineal $\mu = \SI{0.005}{\kilo\gram\over\metre}$ se tensa aplicando una fuerza de \SI{0.25}{\newton}.
El extremo izquierdo se mueve hacia arriba y hacia abajo con un movimiento armónico simple de período \SI{0.5}{\second} y amplitud \SI{0.2}{\metre} mientras se mantiene la tensión constante.
Encontrar:
\begin{enumerate}
	\item La velocidad de la onda generada en la cuerda, la frecuencia y la longitud de onda.
	\item La expresión matemática para el desplazamiento: $y(x,t)$.
	\item La energía cinética media por unidad de longitud, de una partícula del medio.
	\item La energía potencial media por unidad de longitud, de una partícula.
\end{enumerate}
