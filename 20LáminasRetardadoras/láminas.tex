\section*{Láminas retardadoras o de onda}

\item Se hace incidir luz circularmente polarizada en sentido horario sobre una lámina retardadora de cuarto de onda ($+\lambda/4$).
¿Cuál es el estado de polarización de la luz al emerger de la misma?



\item Sobre una lámina de cuarto de onda incide un haz de luz natural de intensidad $I_0$.
¿Con qué estado de polarización emerge?
¿Cuál es su intensidad?
Justifique.



\item Incide luz plano polarizada sobre una lámina de cuarto de onda.
El plano de polarización es paralelo al eje óptico de la misma.
¿Cuál es el estado de polarización de la luz que emerge de la lámina?



\item Una onda linealmente polarizada incide sobre una lámina de media onda ($+\lambda/2$).
El plano de polarización forma un ángulo de \ang{30;;} con el eje óptico de la lámina (considere que el eje óptico es el eje rápido).
¿Cuál es el estado de polarización de la luz que sale de la misma?



\item Incide luz elípticamente polarizada en sentido antihorario sobre una lámina de cuarto de onda.
A medida que se va rotando la lámina retardadora, ¿cuál es el estado de polarización de la luz que emerge?



\item Sobre una lámina de cuarto de onda incide normalmente una vibración monocromática elípticamente polarizada.
Las componentes $E_x$ y $E_y$ del vector campo eléctrico están relacionadas por:
\[
	\frac{ E_x^2 }{ 9 } \pm \frac{ \sqrt{ 2 } }{ 12 } E_x E_y + \frac{ E_y^2 }{ 16 } = \frac{ 1 }{ 2 }
\]
Considere que $x$ es el eje óptico de la lámina, y que dicho eje es el rápido.
\begin{enumerate}
	\item Hallar el estado de polarización de dicha vibración a la salida de la lámina.
	\item Se coloca detrás de la lámina un polarizador cuyo eje óptico forma \ang{30;;} con el eje óptico de la lámina.
	Hallar la vibración que abandona el polarizador, ¿cuál es el porcentaje de energía perdido en la lámina y cuál en el polarizador?
\end{enumerate}
