\section*{Discontinuidades en cuerdas}


\item Nos interesa estudiar la unión de dos cuerdas de distinta densidad lineal $\rho_1$ y $\rho_2$, por lo que las consideraremos semi--infinitas. 
Mientras se las somete a una tensión constante, \(T_0\), incide desde la primera una onda $\psi_i(x,t) = A_i \cos{ \left( k_{1} x- \omega t \right) }$.
\begin{enumerate}
	\item Calcule $k_{1}$ y $k_{2}$, es decir, los números de onda a cada lado de la unión.
	\item Plantee la solución más general para $\psi(x,t)$ de cada lado de la unión.
	\item ¿Qué condiciones deben verificarse en el punto de unión de las cuerdas?
	\item Usando b) y c), calcule la perturbación $\psi(x,t)$ en cada una de las cuerdas.
	\item Determine coeficientes de reflexión, $R$, y transmisión, $T$.
	¿Qué sucede en el caso \(\rho_2 \rightarrow \infty\) y en el \(\rho_1 \rightarrow \rho_2\)? 
\end{enumerate}


\item 
\begin{minipage}[t][2.3cm]{0.6\textwidth}
La cuerda de la izquierda, de densidad \(\rho_1\) y largo \(L\), se encuentra fija en su extremo izquierdo a la pared, y en su extremo derecho a otra cuerda semi-infinita de densidad \(\rho_2\).
Todo el sistema se encuentra sometido a tensión \(T_0\).
Suponga que por la cuerda de densidad \(\rho_2\) incide la onda armónica \(\psi_i(x,t) = A_i \mathrm{e}^{i (\omega t + k_2 x) }\).
\end{minipage}
\begin{minipage}[c][0.5cm][t]{0.34\textwidth}
	\begin{tikzpicture}[scale= 1]
		\draw (-2,-1) -- (-2,1); % pared vertical
		\fill [pattern = north east lines] (-2.25,-1) rectangle (-2,1);
		\draw [ultra thick] (-2,0) -- (0,0) node [midway, above] {\(\rho_1\)};
		\draw [thin] (0,0) -- (3,0) node [midway, above] {\(\rho_2\)};
		\dimline [extension start length= -0.2, extension end length = -0.2] {(-2,-0.4)}{(0,-0.4)}{\(L\)}; 
		\draw [thin, dashed] (3,0) -- (4,0);
	\end{tikzpicture}
\end{minipage}





\item 
Una cuerda de densidad lineal \(\lambda_m\) sometida a una tensión \(T_0\) tiene en su centro,\(x= 0\), un pequeño nudo de masa \(M\).
Este causa que sea parcialmente reflejada viajando en la dirección de las \(x\) positivas dada por \(\psi_i(x,t) = A_i \mathrm{e}^{i ( k x - \omega t ) }\).
Plantee:
\begin{enumerate}
	\item la solución más general para la onda \(\psi (x,t)\) a cada lado del nudo,
	\item y las condiciones de empalme.
	Demuestre que una condición le permite definir que \(A_i + A_r = A_t\) y que la otra implica que \(A_i - A_r = (1 + i \frac{M \omega^2}{k T} )A_t\).
\end{enumerate}
