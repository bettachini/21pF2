\subsection*{Oscilador armónico forzado}
\item Cualquier oscilador armónico \emph{sub-amortiguado} de cierta frecuencia natural \(\omega_0\) tras someterle a un forzado externo y esperar cierto tiempo ajustará su dinámica que responde solo a la forma del forzado.
Siempre la amplitud de la solución homogénea decae pasado un \emph{transitorio}. 

Ya veremos más adelante que cualquier forzado lo podremos descomponer en componentes armónicas, es decir en sumas de términos de senos y cosenos.
Por ahora analizaremos un forzado perféctamente armónico \(F(t) = F_0 \cos(\Omega t)\).

El movimiento resultante tras el \emph{transitorio} responde a la solución particular $x_p(t) = A \sen(\Omega t) + B \cos(\Omega t)$.
Los coeficientes dependerán de la relación entre \(\Omega\) y \(\omega_0\).

\begin{enumerate}
	\item Obtenga expresiones $A(\Omega)$ y $B(\Omega)$.
	\item Grafique $A(\Omega)$ y $B(\Omega)$. ¿Qué sucede con ambas funciones cuando \(\Omega \simeq \omega_0\)? ¿Es justo para la igualdad que esto sucede? ¿Que haría que no fuera así? 
	\item Grafique cualitativamente la posición de la masa en función del tiempo. 
	\item (*) Calcule la potencia media que se consume en el estado estacionario y la potencia media de pérdida por fricción.
	Verifique la igualdad de ambas potencias.
	% \item (*) Verifique que si $x_1(t)$ es solución de la ecuación diferencial cuando la fuerza externa es $F_1(t)$ y $x_2(t)$ cuando la fuerza externa es $F_2(t)$, entonces $x(t) = x_1(t) + x_2(t)$ será solución cuando la fuerza externa sea $F(t) = F_1(t) + F_2(t)$ si y sólo si las condiciones iniciales son la suma de las condiciones iniciales de los dos casos.
	\item (*) Proponga ahora como solución particular la solución compleja \(x_p(t) = A \operatorname{e}^{-i \omega t}\) y explique porque se denoniman así \(A_\text{elástico} = \mathbb{R} (A)\) y \(A_\text{absorbente} = \mathbb{I} (A)\).
\end{enumerate}



%\item Sea un oscilador armónico con una frecuencia de oscilación \(\nu_0= \SI{10}{\hertz}\) y con un tiempo de decaimiento muy largo.
%Si este oscilador es alimentado con una fuerza armónicamente oscilante y con una frecuencia de \SI{10}{\hertz}, adquirirá una gran amplitud, es decir, ``resonará'' en la frecuencia de excitación.
%Ninguna otra fuerza motriz oscilante en forma armónica producirá una gran amplitud (una resonancia). 
%\begin{enumerate}
%	\item Justifique el enunciado anterior. 
%	\item Luego suponga que el oscilador está sujeto a una fuerza que es una pulsación cuadrada repetida periódicamente y cuya duración es \SI{0.01}{\second} repetida una vez por segundo.
%	\item ¿``Resonará'' el oscilador armónico (adquirirá una gran amplitud) bajo la influencia de esta fuerza motriz?
%	\item Suponga que la fuerza motriz es la misma pulsación cuadrada (de ancho \SI{0.01}{\second}) pero repetida dos veces por segundo.
%	¿Resonará el oscilador?
%	Responder a la misma pregunta para velocidades de repetición de \SI{3}{\second} a \SI{9}{\second}.
%\end{enumerate}


