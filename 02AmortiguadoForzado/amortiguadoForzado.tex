\documentclass[11pt,spanish,a4paper]{article}
% Versión 1.er cuat 2021 Víctor Bettachini < bettachini@df.uba.ar >

% Versión 1.er cuat 2021 Víctor Bettachini < bettachini@df.uba.ar >

\usepackage[T1]{fontenc}
\usepackage[utf8]{inputenc}

\usepackage[spanish, es-tabla]{babel}
\def\spanishoptions{argentina} % Was macht dass?
% \usepackage{babelbib}
% \selectbiblanguage{spanish}
% \addto\shorthandsspanish{\spanishdeactivate{~<>}}

\usepackage{graphicx}
\graphicspath{{./figuras/}}
% \usepackage{float}

\usepackage[arrowdel]{physics}
\newcommand{\pvec}[1]{\vec{#1}\mkern2mu\vphantom{#1}}
% \usepackage{units}
\usepackage[separate-uncertainty=true, multi-part-units=single, locale=FR]{siunitx}
\usepackage{isotope} % $\isotope[A][Z]{X}\to\isotope[A-4][Z-2]{Y}+\isotope[4][2]{\alpha}

\usepackage{tasks}
\usepackage[inline]{enumitem}
% \usepackage{enumerate}

\usepackage{hyperref}

% \usepackage{amsmath}
% \usepackage{amstext}
\usepackage{amssymb}

\usepackage{tikz}
\usepackage{tikz-dimline}
\usetikzlibrary{math}
\usetikzlibrary{arrows.meta}
% \usetikzlibrary{snakes}
% \usetikzlibrary{calc}
\usetikzlibrary{decorations.pathmorphing}
\usetikzlibrary{patterns}

\usepackage[hmargin=1cm,vmargin=1.6cm,nohead]{geometry}
% \voffset-3.5cm
% \hoffset-3cm
% \setlength{\textwidth}{17.5cm}
% \setlength{\textheight}{27cm}

\usepackage{lastpage}
\usepackage{fancyhdr}
\pagestyle{fancyplain}
\fancyhead{}
\fancyfoot{{\tiny \textcopyright DF, FCEyN, UBA}}
\fancyfoot[C]{ {\tiny Actualizado al \today} }
\fancyfoot[RO, LE]{Pág. \thepage/\pageref{LastPage}}
\renewcommand{\headrulewidth}{0pt}
\renewcommand{\footrulewidth}{0pt}


\begin{document}
\begin{center}
\textbf{Física 2} (Físicos) \hfill \textcopyright {\tt DF, FCEyN, UBA}\\
%	\textsc{\large Física 2 (Físicos)} - Prof. Diana Skigin - 2"o cuat. 2014\\
%	\textsc{\large Primer Cuatrimestre - 2014}\\
\textsc{\LARGE Oscilador armónico amortiguado y forzado}\\
\end{center}


\begin{enumerate}


\subsection*{Oscilador armónico amortiguado}

\item Considere el movimiento de una masa $m$ sujeta a un resorte de constante elástica $K= m \omega_0^2$ y constante de amortiguamiento por unidad de masa $\Gamma$. \\
Demuestre que el resultado para el oscilador ``sobreamortiguado'' dado por
\[
x(t)=\mathrm{e}^{-\Gamma t/2}\left\{ x(0)\cosh\left(\left|\omega\right|t\right)+\left[\dot{x}(0)+\frac{1}{2}\Gamma x(0)\right]\frac{\senh\left(\left|\omega\right|t\right)}{\left|\omega\right|}\right\} 
\]
se deduce de las siguientes
\[
x(t)=\mathrm{e}^{-\Gamma t/2}\left\{ x(0)\cos\left(\omega t\right)+\left[\dot{x}(0)+\frac{1}{2}\Gamma x(0)\right]\frac{\sen\left(\omega t\right)}{\omega}\right\} 
\]
\[
\omega=\pm i\left|\omega\right|,\,\,\,\left|\omega\right|=\sqrt{\frac{1}{4}\Gamma^{2}-\omega_{0}^{2}}
\]
\textbf{Sugerencia}: verifique las identidades $\cos\left(ix\right)=\cosh\left(x\right)$ y $\sen\left(ix\right)=i\senh\left(x\right)$; luego úselas.



\item Comenzando con la ecuación general dada en el problema anterior para oscilaciones libres subamortiguadas, muestre que para amortiguamiento crítico la solución es:
\[
x(t)=\mathrm{e}^{-\Gamma t/2}\left\{ x(0)+\left[\dot{x}(0)+\frac{1}{2}\Gamma x(0)\right]t\right\} 
\]
Muestre que también se obtiene este resultado comenzando con la ecuación para oscilaciones sobreamortiguadas.



\subsection*{Oscilador armónico forzado}

\item% \quad{}
\begin{enumerate}
	\item Escriba la ecuación de movimiento para una masa $m$ sujeta a un resorte de constante elástica $k$ y constante de amortiguamiento por unidad de masa $\Gamma$, sobre la que se realiza una fuerza dependiente del tiempo $F(t)$. 
	\item Proponga la siguiente solución homogénea: $x_{h}(t)=C\mathrm{e}^{-t/2\tau}\cos\left(\omega_{1}t+\theta\right)$ y halle los valores de $\tau$ y de $\omega_{1}$.
	¿De qué depende el valor de $C$ y de $\theta$?
	¿Es lícito plantear las condiciones iniciales sobre la solución homogénea? 
 \item Considere que $F(t)$ tiene la forma $F(t)=F_{0}\cos\left(\Omega t\right)$ (discuta si se pierde generalidad al suponer que la fuerza externa tiene esa forma) y proponga la siguiente solución particular: $x_{p}(t)=A\sen\left(\Omega t\right)+B\cos\left(\Omega t\right)$.
Obtenga $A$ y $B$. Grafique cualitativamente $A$ y $B$ en función de $\omega$. 
	\item Grafique cualitativamente la posición de la masa en función del tiempo. 
	\item Calcule la potencia media que se consume en el estado estacionario y la potencia media de pérdida por fricción.
	Verifique la igualdad de ambas potencias. 
 \item Verifique que si $x_1(t)$ es solución de la ecuación diferencial cuando la fuerza externa es $F_1(t)$ y $x_2(t)$ lo es cuando la fuerza externa es $F_2(t)$, entonces $x(t)=x_1(t)+x_2(t)$ será solución de la ecuación diferencial cuando la fuerza externa sea $F(t)=F_1(t)+F_2(t)$ si y sólo si las condiciones iniciales son la suma de las condiciones iniciales de los dos casos. 
	\item Proponga ahora como solución particular la solución compleja $x_p(t)=A \mathrm{e}^{-i \omega t}$ y demuestre que $\Re (A)=A_\text{elástico}$ y que $\Im (A)= A_\mathrm{absorbente}$.
	¿Por qué es así?
\end{enumerate}



\item Sea un oscilador armónico con una frecuencia de oscilación \(\nu_0= \SI{10}{\hertz}\) y con un tiempo de decaimiento muy largo.
Si este oscilador es alimentado con una fuerza armónicamente oscilante y con una frecuencia de \SI{10}{\hertz}, adquirirá una gran amplitud, es decir, ``resonará'' en la frecuencia de excitación.
Ninguna otra fuerza motriz oscilante en forma armónica producirá una gran amplitud (una resonancia). 
\begin{enumerate}
	\item Justifique el enunciado anterior. 
	\item Luego suponga que el oscilador está sujeto a una fuerza que es una pulsación cuadrada repetida periódicamente y cuya duración es \SI{0.01}{\second} repetida una vez por segundo.
	\item ¿``Resonará'' el oscilador armónico (adquirirá una gran amplitud) bajo la influencia de esta fuerza motriz?
	\item Suponga que la fuerza motriz es la misma pulsación cuadrada (de ancho \SI{0.01}{\second}) pero repetida dos veces por segundo.
	¿Resonará el oscilador?
	Responder a la misma pregunta para velocidades de repetición de \SI{3}{\second} a \SI{9}{\second}.
\end{enumerate}


\end{enumerate}

\end{document}
