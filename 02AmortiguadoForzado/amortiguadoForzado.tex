\documentclass[11pt,spanish,a4paper]{article}
% Versión 1.er cuat 2021 Víctor Bettachini < bettachini@df.uba.ar >

% Versión 1.er cuat 2021 Víctor Bettachini < bettachini@df.uba.ar >

\usepackage[T1]{fontenc}
\usepackage[utf8]{inputenc}

\usepackage[spanish, es-tabla]{babel}
\def\spanishoptions{argentina} % Was macht dass?
% \usepackage{babelbib}
% \selectbiblanguage{spanish}
% \addto\shorthandsspanish{\spanishdeactivate{~<>}}

\usepackage{graphicx}
\graphicspath{{./figuras/}}
% \usepackage{float}

\usepackage[arrowdel]{physics}
\newcommand{\pvec}[1]{\vec{#1}\mkern2mu\vphantom{#1}}
% \usepackage{units}
\usepackage[separate-uncertainty=true, multi-part-units=single, locale=FR]{siunitx}
\usepackage{isotope} % $\isotope[A][Z]{X}\to\isotope[A-4][Z-2]{Y}+\isotope[4][2]{\alpha}

\usepackage{tasks}
\usepackage[inline]{enumitem}
% \usepackage{enumerate}

\usepackage{hyperref}

% \usepackage{amsmath}
% \usepackage{amstext}
\usepackage{amssymb}

\usepackage{tikz}
\usepackage{tikz-dimline}
\usetikzlibrary{math}
\usetikzlibrary{arrows.meta}
% \usetikzlibrary{snakes}
% \usetikzlibrary{calc}
\usetikzlibrary{decorations.pathmorphing}
\usetikzlibrary{patterns}

\usepackage[hmargin=1cm,vmargin=1.6cm,nohead]{geometry}
% \voffset-3.5cm
% \hoffset-3cm
% \setlength{\textwidth}{17.5cm}
% \setlength{\textheight}{27cm}

\usepackage{lastpage}
\usepackage{fancyhdr}
\pagestyle{fancyplain}
\fancyhead{}
\fancyfoot{{\tiny \textcopyright DF, FCEyN, UBA}}
\fancyfoot[C]{ {\tiny Actualizado al \today} }
\fancyfoot[RO, LE]{Pág. \thepage/\pageref{LastPage}}
\renewcommand{\headrulewidth}{0pt}
\renewcommand{\footrulewidth}{0pt}


\begin{document}
\begin{center}
\textbf{Física 2} (Físicos) \hfill \textcopyright {\tt DF, FCEyN, UBA}\\
%	\textsc{\large Física 2 (Físicos)} - Prof. Diana Skigin - 2"o cuat. 2014\\
%	\textsc{\large Primer Cuatrimestre - 2014}\\
\textsc{\LARGE Oscilador armónico amortiguado y forzado}\\
\end{center}


\begin{enumerate}


\subsection*{Oscilador armónico amortiguado}

\item Una pesa de masa $m$ está sujeta a un resorte de constante elástica $k$, por lo que la frecuencia natural de oscilación es $\omega_0 = \sqrt{ \frac{k}{m} }$.
Actúa en este sistema un amortiguador que provee una amortiguación lineal con la velocidad de constante de amortiguamiento \(c\) que por unidad de masa es $\Gamma= c/m$.
	\begin{enumerate}
	\item Proponga la siguiente solución homogénea: $x_h(t) = C\mathrm{e}^{-t/2\tau}\cos(\omega_1 t + \theta )$ y halle los valores de $\tau$ y de $\omega_1$.
	¿De qué depende los valores de $C$ y $\theta$?
	¿Porqué no es lícito imponer las condiciones iniciales a la solución homogénea?
	\item Repase las condiciones de \(\Gamma\) y \(\omega_0\) en que se obtienen soluciones:
	\begin{itemize}
		\item sub-amortiguadas,
		\item críticamente amortiguadas, y
		\item sobre-amortiguadas,
	\end{itemize} \label{subamortiguado}
	graficando \(x(t)\) para distintos valores de estos parámetros.
	\item Verifique que la solución general para el oscilador libre \emph{sobre-amortiguado}
	\[
		x(t) = \operatorname{e}^{-\Gamma t/2} \left\{ x(0) \cosh \left( \tilde{\Omega}  t \right) + \left[ \dot{x}(0) + \frac{1}{2} \Gamma x(0) \right] \frac{ \senh \left( \tilde{\Omega} t \right) }{ \tilde{\Omega} } \right\} 
	\]
	puede obtenerse a partir de esta para el \emph{sub-amortiguado}
	\[
		x(t) = \operatorname{e}^{-\Gamma t/2} \left\{ x(0) \cos \left( \tilde{\omega} t\right) + \left[ \dot{x}(0) + \frac{1}{2} \Gamma x(0) \right] \frac{ \sen \left( \tilde{\omega} t \right)}{ \tilde{\omega} }\right\}, 
	\]
	donde \(
		\tilde{\omega} = \pm i \tilde{\Omega}
	\), \(
		\tilde{\Omega} = \sqrt{\frac{1}{4}\Gamma^{2}-\omega_{0}^{2}}.
	\)
	\textbf{Sugerencia}: verifique y aproveche las identidades $\cos\left(ix\right)=\cosh\left(x\right)$ y $\sen\left(ix\right)=i\senh\left(x\right)$.
	\item Escriba la expresión de la trayectoria \(x(t)\) y calcule la energía en \(x(0)\) para la condición inicial \(x(0)= x_0\) que parte del reposo, es decir \(\dot{x}(0) = 0\). 
	\item A partir de la solución general para el oscilador libre sub-amortiguado, muestre que la solucío para el amortiguamiento crítico es
	\[
		x(t)=\mathrm{e}^{-\Gamma t/2}\left\{ x(0)+\left[\dot{x}(0)+\frac{1}{2}\Gamma x(0)\right]t\right\} .
		\]
	Verifique que también podría haberle obtenido a partir de la solucíon para oscilaciones sobre-amortiguadas.
\end{enumerate}



\subsection*{Oscilador armónico forzado}
Sabemos que cualquier oscilador armónico sub-amortiguado sometido a un forzado externo transcurrido cierto tiempo la amplitud de la solución homogénea decae.
Esto es, pasado el \emph{transitorio} solo sobrevive la solución particular que responde al forzado.

Ya veremos en el curso que cualquier fuerza externa \(F(t)\) la podremos descomponer en componentes armónicas, es decir en sumas de términos de senos y cosenos.
Por ahora estudiaremos un oscilador sub-amortiguado, como el presentado en \ref{subamortiguado}, al que se somete a un forzado perféctamente armónico \(F(t) = F_0 \cos(\Omega t)\).

El movimiento resultante, pasado el \emph{transitorio}, puede representarse por $x_p(t) = A \sen(\Omega t) + B \cos(\Omega t)$, por lo que esta será nuestra solución particular.
Los coeficientes de este seno y coseno dependerán de \(\Omega\), es decir serán funciones $A(\Omega)$ y $B(\Omega)$.

\begin{enumerate}
	\item Proponga la solución particular: 
	Obtenga expresiones para $A(\Omega)$ y $B(\Omega)$.
	\item A partir Grafique $A(\Omega)$ y $B(\Omega)$.
 \item Considere que $F(t)$ tiene la forma $F(t)=F_{0}\cos\left(\Omega t\right)$ (discuta si se pierde generalidad al suponer que la fuerza externa tiene esa forma) y proponga la siguiente solución particular: $x_{p}(t)=A\sen\left(\Omega t\right)+B\cos\left(\Omega t\right)$.
	\item Grafique cualitativamente la posición de la masa en función del tiempo. 
	\item Calcule la potencia media que se consume en el estado estacionario y la potencia media de pérdida por fricción.
	Verifique la igualdad de ambas potencias. 
 \item Verifique que si $x_1(t)$ es solución de la ecuación diferencial cuando la fuerza externa es $F_1(t)$ y $x_2(t)$ lo es cuando la fuerza externa es $F_2(t)$, entonces $x(t)=x_1(t)+x_2(t)$ será solución de la ecuación diferencial cuando la fuerza externa sea $F(t)=F_1(t)+F_2(t)$ si y sólo si las condiciones iniciales son la suma de las condiciones iniciales de los dos casos. 
	\item Proponga ahora como solución particular la solución compleja $x_p(t)=A \mathrm{e}^{-i \omega t}$ y demuestre que $\Re (A)=A_\text{elástico}$ y que $\Im (A)= A_\mathrm{absorbente}$.
	¿Por qué es así?
\end{enumerate}



%\item Sea un oscilador armónico con una frecuencia de oscilación \(\nu_0= \SI{10}{\hertz}\) y con un tiempo de decaimiento muy largo.
%Si este oscilador es alimentado con una fuerza armónicamente oscilante y con una frecuencia de \SI{10}{\hertz}, adquirirá una gran amplitud, es decir, ``resonará'' en la frecuencia de excitación.
%Ninguna otra fuerza motriz oscilante en forma armónica producirá una gran amplitud (una resonancia). 
%\begin{enumerate}
%	\item Justifique el enunciado anterior. 
%	\item Luego suponga que el oscilador está sujeto a una fuerza que es una pulsación cuadrada repetida periódicamente y cuya duración es \SI{0.01}{\second} repetida una vez por segundo.
%	\item ¿``Resonará'' el oscilador armónico (adquirirá una gran amplitud) bajo la influencia de esta fuerza motriz?
%	\item Suponga que la fuerza motriz es la misma pulsación cuadrada (de ancho \SI{0.01}{\second}) pero repetida dos veces por segundo.
%	¿Resonará el oscilador?
%	Responder a la misma pregunta para velocidades de repetición de \SI{3}{\second} a \SI{9}{\second}.
%\end{enumerate}


\end{enumerate}

\end{document}
