\section*{Dos lentes}


\item Una lente delgada convergente, de distancia focal \SI{30}{\centi\metre}, se coloca \SI{20}{\centi\metre} a la izquierda de otra lente delgada divergente de distancia focal \SI{50}{\centi\metre}.
Para un objeto colocado a \SI{40}{\centi\metre} a la izquierda de la primera lente determine la imagen final.
¿Cuál es el aumento?
La imagen, ¿es real o virtual, es directa o invertida?



\item Una lente delgada convergente de \SI{5}{\centi\metre} de diámetro y \SI{4}{\centi\metre} de distancia focal se halla \SI{2}{\centi\metre} a la derecha de un diafragma de \SI{3}{\centi\metre} de diámetro.
\begin{enumerate}
	\item Si se coloca un objeto puntual axial a \SI{9}{\centi\metre} a la izquierda de la lente, determinar la posición y el tamaño de las pupilas de entrada y salida, en forma gráfica y analítica.
	\item Se desplaza al objeto, perpendicularmente al eje óptico, una distancia de \SI{1.5}{\centi\metre}.
Determine en forma gráfica y analítica, si el diafragma de apertura está bien definido.
	\item Repita el cálculo para el caso en que el objeto es desplazado \SI{3}{\centi\metre}. 
Discuta si hay o no \emph{vigneteo}, y en caso de haberlo calcule la máxima altura del objeto para que el diafragma de apertura esté bien definido.
\end{enumerate}
