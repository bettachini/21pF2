\section*{Dispositivos con dos o más lentes}


\item Un microscopio consta de un objetivo de \SI{4}{\milli\metre} de distancia focal y de un ocular de \SI{30}{\milli\metre} de distancia focal.
La distancia entre el foco imagen del objetivo y el foco objeto del ocular es \SI{18}{\centi\metre}.
Calcule:
\begin{enumerate}
	\item El aumento normal del microscopio.
	\item La distancia objeto--objetivo.
	\item Sabiendo que el microscopio no cuenta con diafragmas adicionales, y que la pupila de salida debe ser real, y del mismo diámetro aproximado que la pupila del ojo (\(\approx \SI{12}{\milli\metre}\)), discuta cuál de las dos lentes debe ser el diafragma de apertura, cuál debe ser su diámetro y en qué posición se halla la pupila de salida.
	\item Discuta en qué posiciones colocaría un diafragma de campo, y si esta introducción modifica o no la determinación del diafragma de apertura.
	Justifique claramente sus respuestas.
\end{enumerate}



\item Un anteojo astronómico (telescopio refractor o de Galileo, configuración de conjugados infinitos) utiliza como objetivo una lente convergente de \SI{2}{\metre} de distancia focal y \SI{10}{\centi\metre} de diámetro, y como ocular una lente convergente de \SI{4}{\centi\metre} de distancia focal. 
Determine:
\begin{enumerate}
	\item El aumento eficaz.
	\item Las características de la primer imagen de la Luna y de la imagen final.
	El ancho de la Luna cubre \ang{;31;08} del campo visual humano promedio.
	\item El largo total del tubo.
	\item (*) El mínimo diámetro del ocular para que el objetivo sea diafragma de apertura.
	Recordar que la luna no es puntual, y por ende hay puntos objeto extra-axiales.
	\item Suponiendo que el diámetro del ocular sea de \SI{4}{\centi\metre}, la posición y el tamaño de la pupila de salida.
	\item (*) La posición en que debe colocarse el ojo.
	\item (*) La posición en que debe colocarse, de ser posible, un diafragma de campo.
	\item (*) El mínimo diámetro del posible diafragma de campo para que la imagen de la Luna se vea completa.
\end{enumerate}


\item Delante del objetivo de un telescopio refractor de dos lentes y a distancia $s > f'_{obj}$ se coloca un objeto de altura $h$.
\begin{enumerate}
	\item Obtenga la posición de la imagen en función de $f'_{obj}$, $f'_{oc}$ y $s$. 
	\item Calcule el tamaño de la imagen y demuestre gráfica y analíticamente que el aumento lateral es independiente de la posición del objeto.
\end{enumerate}


\item (*) Un sensor fotográfico \emph{CMOS} de uso científico tiene por dimensiones \(\SI{35}{\milli\metre} \times \SI{124}{\milli\metre}\) y por objetivo una lente convergente de \SI{50}{\milli\metre} ($f'$).
\begin{enumerate}
	\item Si se quiere fotografiar objetos que disten del objetivo entre \SI{1}{\metre} e infinito, ¿qué longitud debe tener la rosca que lo mueve?
	\item El sistema se halla enfocando sobre la película a un objeto distante \SI{1}{\metre}.
	Analice qué ocurre con la profundidad de campo para objetos distantes del primero \SI{20}{\centi\metre}.
	Repita el análisis si el objeto enfocado inicialmente se hallase a \SI{3}{\metre} y a \SI{10}{\metre}.
	\item Discuta qué sucedería con la longitud de rosca, la profundidad de campo y la aproximación paraxial si se quisiera fotografiar objetos distantes \SI{50}{\centi\metre}.
	\item Teniendo en cuenta que el sensor es el diafragma de campo, discuta los posibles ángulos de campo máximos.
	Calcule los ángulos de campo asociados a la mayor dimensión del sensor, para un objeto que se encuentra en el infinito.
	¿Cómo varían los ángulos de campo con la posición del objeto?
	¿Cuánto es posible desplazar el objeto para que la variación no supere el 5\%?
	\item Si se quiere fotografiar un árbol de 5 m de altura, y se lo quiere fotografiar entero, ¿cuál es la mínima distancia a la que hay que ponerse?
	\item Sabiendo que las aperturas inversas, $f/D$ siendo \(D\) la dimensión del diafragma, varían ente \num{1.4} y \num{22}, calcule los tamaños máximo y mínimo del diafragma.
	Discuta cualitativamente el porqué y cómo de las variaciones de tamaño del diafragma (su relación con la velocidad del objeto, con la de obturación, con la luminosidad ambiente, etc.)
\end{enumerate}
