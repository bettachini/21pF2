\section*{Espejos curvos}

\item
\begin{enumerate}
	\item (*) Partiendo de la ecuación de las dioptras obtenga la ecuación de los espejos esféricos. 
	\item ¿Cómo se modifica la distancia focal de un espejo esférico si se lo sumerge en agua?
	\item Un espejo esférico cóncavo produce una imagen cuyo tamaño es el doble del tamaño del objeto, siendo la distancia objeto--imagen de \SI{15}{\centi\metre}.
	Calcule la distancia focal del espejo.
\end{enumerate}



\item (*) Una esfera maciza de radio $R$ e índice de refracción \num{1.5} ha sido espejada en una mitad de su superficie.
Se coloca un objeto sobre el eje de la esfera a distancia $2R$ del vértice de la semiesfera no espejada.
Hallar:
\begin{enumerate}
\item La imagen final, en forma analítica, luego de todas las refracciones
y reflexiones que hayan tenido lugar.
\item El aumento y las características de la imagen final.
\item Ídem (a) mediante trazado de rayos.
\end{enumerate}
