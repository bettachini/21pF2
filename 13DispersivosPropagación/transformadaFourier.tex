\section*{Transformada de Fourier}

\item 
\textbf{Relación entre ancho de un paquete y el desfasaje de las frecuencias que lo componen}
\begin{enumerate}
	\item Tome el siguiente pulso con un espectro Gaussiano de ancho $\Delta k$ centrado en $k_0$ (note que las frecuencias están en fase):
$$
F(k)=A \operatorname{e}^{-\frac{ ( k - k_0 )^2 }{ 4 \Delta k^2 } }.
$$
Calcule $f(x) = \mathcal{F}^{-1}[F(k)]$ y vea que tiene una envolvente Gaussiana que modula una portadora de frecuencia $k_{0}$.
Note que el pulso está centrado en $x=0$ y que se cumple la relación $\Delta x \Delta k = 1/2$ (el paquete Gaussiano es el de mínima incerteza).
Ayuda: \( \int\limits_{-\infty}^{\infty} \operatorname{e}^{-\alpha x^2} \dd{x} = \sqrt{\frac{\pi}{\alpha} } \), \( \int\limits_{-\infty}^{\infty} \operatorname{e}^{-(x+ a)^2} \dd{x} = \sqrt{\pi}\).
	\item Ahora desfase las distintas frecuencias en forma lineal, tal que:
$$
F(k)=A \operatorname{e}^{ -\frac{ ( k - k_0 )^2 }{ 4 \Delta k^2 } } \operatorname{e}^{ i \alpha (k - k_0 ) }.
$$
Calcule $f(x)$ y vea que es el mismo pulso que en la parte a), pero desplazado en $\alpha$ hacia la derecha (una fase lineal sólo corre la función).
	\item Ahora agregue una fase cuadrática, es decir:
$$
F(k) = A \operatorname{e}^{-\frac{(k-k_{0})^{2}}{4\Delta k^{2}} } \operatorname{e}^{i \beta ( k - k_0 )^2 }.
$$
Calcule $f(x)$ y vea que es un pulso Gaussiano centrado en $x=0$ pero con un ancho $\Delta x$ que cumple:
$$
\Delta x \Delta k = \frac{1}{2} \sqrt{ 1 + 16 \beta^2 \Delta k^4 }.
$$
¿Es cierto que si se quiere disminuir el ancho de un paquete siempre se debe aumentar $\Delta k$?
Derive $\Delta x$ con respecto a $\Delta k$ de la expresión anterior y analice lo pedido.
\end{enumerate}



\item \textbf{Conjugación de la transformada de Fourier}\\
Muestre que si $\phi(t) \in \mathcal{R}$ y $\psi(\omega)= \mathcal{F} \left[ \phi (t) \right]$ es su transformada de Fourier, esta última cumple que \( \overline{\psi}(\omega) = \psi(- \omega) \), es decir, que para obtener su conjugada de la transformada basta con invertir el signo de $\omega$.
Aproveche esto para escribir $\phi(t)$ como superposición de senos y cosenos.
	% \item Muestre que su transformada de Fourier $\psi(\omega)$ cumple $\psi(\omega)=\psi(-\omega)$.
	% cumple $\psi(\omega) = \psi^*(-\omega)$ ($\psi(\omega) = |\psi(-\omega)|$).
