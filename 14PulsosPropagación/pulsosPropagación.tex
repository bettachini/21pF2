% \section*{Pulsos y espectros cuadrados}

\item
\textbf{Espectro plano acotado}\\ 
\(\hat{\phi }(\omega)\) en un $\Delta\omega$ centrado en un $\omega_0$ presenta una amplitud constante $\frac{1}{\Delta \omega}$. 
En otras $\omega$ es nula.
\begin{enumerate}
	\item
	Verifique que el correspondiente $\phi(t) = \mathcal{F}^{-1} \hat{\phi }(\omega)$ está dado por:
	$$
		\phi(t)
		= \frac{1}{2 \pi} \left[ \frac{ \sen{ \left( \frac{\Delta \omega}{2} t \right) } }{\frac{\Delta \omega}{2} t} \right] \operatorname{e}^{i \omega_0 t}
		= \frac{1}{2 \pi} \operatorname{senc} \left( \frac{\Delta \omega}{2} t \right) \operatorname{e}^{i \omega_0 t}.
%		= \frac{1}{\sqrt{2 \pi}} \left[ \frac{ \sen{ \left( \frac{\Delta \omega}{2} t \right) } }{\frac{\Delta \omega}{2} t} \right] \operatorname{e}^{i \omega_0 t}
%		= \frac{1}{\sqrt{2 \pi}} \operatorname{senc} \left( \frac{\Delta \omega}{2} t \right) \operatorname{e}^{i \omega_0 t}.
	$$
	\item
	Grafique $\hat{\phi }(\omega)$ y $\left|\phi(t)\right|$ para $\omega_0 = \SI{100}{\per\second}$ y $\Delta \omega = \SI{4}{\per\second}$.
	¿Qué parte de la expresión de $\phi(t)$ impone límites temporales para el paquete de mayor amplitud?
	Determine tales límites.
	\item 
	Sea $T$ un intervalo de tiempo más prolongado que la duración de cualquier experimento que pueda idear.
	Muestre que si $\Delta \omega$ es suficientemente pequeño como para que $\Delta \omega T \ll 1$, entonces durante un tiempo menor que $T$, $\phi(t)$ es una función armónica de amplitud y fase casi constantes.
\end{enumerate}



\item \textbf{Tren de pulsos cuadrados}
\begin{enumerate}
	\item Muestre que $\mathcal{F}$ es lineal, por tanto
	$
	\mathcal{F} \left( a f(x) + b g(x) \right) = a \mathcal{F} f(x) + b \mathcal{F} g(x),
	$
	donde $a$, $b$ son constantes.

	\item
	\begin{minipage}[t][5cm]{0.43\textwidth}
	\(\phi(t)\) es una serie de pulsos cuadrados de duración \(\Delta t\) que se presentan $N$ veces con un período $\tau$ (\(\Delta t < \tau\)).
	Si \(f(n,t)\) describe la función en cualquiera de los intervalos \( [n \tau, (n+1) \tau] \) que contiene estos pulsos de amplitud no nula $\phi_0$ en $[n\tau, n \tau + \Delta t]$ de forma que \(\phi(t)= \sum_{n=0}^{N-1} f(n,t)\), compruebe que 
	$$
	\hat{\phi}(\nu) = \mathcal{F} \phi(t) = \mathcal{F} \left[ \sum_{n=0}^{N-1} f(n,t) \right] = \sum_{n=0}^{N-1} \operatorname{e}^{ - i n \omega \tau} \mathcal{F} f(0,t).
	$$
	\end{minipage}
	\begin{minipage}[c][0.5cm][t]{0.5\textwidth}
		\begin{tikzpicture}
%			\tikzmath{
%			\Dt= 1;
%			\ttau= 2;
%			\haute= 1;
%			}
			\def \Dt{1};
			\def \ttau{2};
			\def \haute{1};
			\draw [-Latex] (0,0) -- (8,0) node [anchor=north] {\( t \)};	% eje x
			\draw [Latex-](0,2*\haute) node [anchor=west] {\( \phi(t) \)} -- (0,0) node [anchor=north] {\(0\)};	% eje y
			\draw [thin, dashed] (0,\haute) node [anchor=east] {\( \phi_0 \)} -- (0,\haute);
			\draw [ultra thick] (0,\haute) -- (\Dt,\haute);	% cuerda
			\draw [thin, dashed] (\Dt,0) node [anchor=north] {\( \Delta t \)} -- (\Dt,\haute);
			\draw [ultra thick] (\Dt,0) -- (\ttau+\Dt,0);	% cuerda
			\draw [thin, dashed] (\Dt+\ttau,0) node [anchor=north] {\( \tau  \)} -- (\ttau+\Dt,\haute);
			\draw [ultra thick] (\Dt+\ttau,\haute) -- (\Dt+\ttau+\Dt,\haute);	% cuerda
			\draw [thin, dashed] (\Dt+\ttau+\Dt,0) node [anchor=north] {\( \tau+ \Delta t \)} -- (\Dt+\ttau+\Dt,\haute);
			\draw [ultra thick] (\Dt+\ttau+\Dt,0) -- (\Dt+\ttau+\Dt+\ttau,0);	% cuerda
			\draw [thin, dashed] (\Dt+\ttau+\Dt+\ttau,0) node [anchor=north] {\( 2\tau \)} -- (\Dt+\ttau+\Dt+\ttau,\haute);
			\draw [ultra thick] (\Dt+\ttau+\Dt+\ttau,\haute) -- (\Dt+\ttau+\Dt+\ttau+\Dt,\haute);	% cuerda
			\draw [thin, dashed] (\Dt+\ttau+\Dt+\ttau+\Dt,0) node [anchor=north] {\( 2\tau+ \Delta t \)} -- (\Dt+\ttau+\Dt+\ttau+\Dt,\haute);
		\end{tikzpicture}
	\end{minipage}
	\item Resuelva \(\mathcal{F} f(0,t)\) para obtener la expresión completa de $\hat{\phi }(\nu) = \mathcal{F} \phi(t)$.

	\item El rasgo más prominente de \(\hat{\phi }(\nu)\) son picos en \(\nu_p = p \nu_1 \; (p \in \mathbb{N})\) donde \(\nu_1 = \frac{1}{\tau}\), es decir, una serie de armónicos de \(\nu_1\).
	Encuentre en la expresión de \(\hat{\phi }(\nu)\) el término que depende de \(\tau\) responsable de este comportamiento y verifique \(\nu_p\). 
	
	\item De similar análisis identifique qué término con dependencia en \(\Delta t\) hace que los armónicos más importantes se detecten en \(0 < \nu < \frac{1}{\Delta t}\).

	\item Compruebe también que el ancho de banda de los armónicos es \(\delta \nu = \frac{2}{N \tau}\), y calcule cuánto más pequeño es que el $\Delta \nu$ entre sucesivos $\nu_p$.
	% \item Muestre que para un valor finito de $T_\text{largo}$ el análisis de Fourier de esta pulsación cuadrada repetida casi periódicamente, consiste en una superposición de armónicos casi discretos de la frecuencia fundamental $\nu_{1}=1/T_{1}$, siendo realmente cada armónico un continuo de frecuencias que se extiende sobre una banda de ancho $\delta \nu \approx 1/ T_\text{largo}$.
	% \item ¿Por qué $\Delta t \Delta \nu \approx 1$ si, en principio, podría $\Delta t \Delta \nu \gg 1$?
	% La misma pregunta para $\delta \nu$ y $T_\text{largo}$.
\end{enumerate}



\item 
(*) \textbf{Interfaz entre medios no dispersivos}\\
\begin{minipage}[t][2.1cm]{0.61\textwidth}
Dos cuerdas semi-infinitas de distinta densidad lineal de masa, $\lambda_{m\,\text{izq}}$ y $\lambda_{m\,\text{der}}$, están unidas y sometidas a una tensión $T_0$.
Sobre la primera se propaga hacia $+\hat{x}$ la perturbación que muestra la figura.
Se conocen $\lambda_{m\,\text{izq}}$, $\lambda_{m\,\text{der}}$, $T_0$, $x_0$, $\Delta x$ y $h$, y se considera que los medios son no dispersivos.
\end{minipage}
\begin{minipage}[c][0.4cm][t]{0.34\textwidth}
	\begin{tikzpicture}
%		\tikzmath{
%			\Deltax = 1;
%			\x0 = -1;
%			\haute = 1;
%			\xmin = -2.5;
%			\xmax = -\xmin;
%			\ymin = 0;
%			\ymax = 2;
%		}
		\def \Deltax{1};
		\def \x0{-1};
		\def \haute{1};
		\def \xmin{-3};
		\def \xmax{2.5};
		\def \ymin{0};
		\def \ymax{2};
		\coordinate (quiebreIzq) at ({(\x0-\Deltax/2)},0);
		\coordinate (quiebreSuperior) at ({\x0},\haute);
		\coordinate (quiebreDer) at ({(\x0+\Deltax/2)},0);
		\draw [-Latex, thin] (\xmin,0) -- (\xmax+0.3,0) node [anchor=north] {\(x\) [m]};	% eje x
		\draw [Latex-, thin](0,\ymax) node [anchor=west] {\( \psi(x,0) \)} -- (0,\ymin) node [anchor=north] {\(0\)};	% eje y
		\draw [ultra thick] (\xmin+0.3,0) -- (quiebreIzq) node [near start, above] {\(\lambda_{m\,\mathrm{izq}}\)} -- (quiebreSuperior) -- (quiebreDer) -- (0,0) ; %	% cuerda izq
		% \draw [ultra thick] (\xmin+0.3,0) -- (quiebreIzq) -- (quiebreSuperior) -- (quiebreDer) -- (0,0) node [near start, above] {\(\lambda_{m\,\mathrm{izq}}\)}; %	% cuerda izq
		\draw [ultra thick, gray] (0,0) -- ({\xmax-.3},0) node [near end, above] {\(\lambda_{m\,\mathrm{der}}\)};	% cuerda der
		\draw [thin, dashed] (0,\haute) node [anchor = west] {\( h \)} -- (quiebreSuperior);
		\dimline [label style= {above=0}, extension start length=-0.6, extension end length=-0.6]{(-1.5,-.6)}{(-0.5,-.6)}{\( \Delta x \)};
		\dimline [label style= {above=0}, extension start length=0.4, extension end length=0.4]{($(quiebreSuperior) + (0, 0.4)$)}{(0, \haute + 0.4)}{\( x_0 \)};
	\end{tikzpicture}
% 	\includegraphics[width=\textwidth]{ej2-20}
\end{minipage}
\begin{enumerate}
	\item Hallar el desplazamiento $\psi(x,t)$.
	\item Explique cualitativamente cómo cambian estos resultados si el medio es dispersivo.
\end{enumerate}


